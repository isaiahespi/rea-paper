% Options for packages loaded elsewhere
\PassOptionsToPackage{unicode}{hyperref}
\PassOptionsToPackage{hyphens}{url}
\PassOptionsToPackage{dvipsnames,svgnames,x11names}{xcolor}
%
\documentclass[
  11pt,
  a4paper,
]{scrartcl}

\usepackage{amsmath,amssymb}
\usepackage{setspace}
\usepackage{iftex}
\ifPDFTeX
  \usepackage[T1]{fontenc}
  \usepackage[utf8]{inputenc}
  \usepackage{textcomp} % provide euro and other symbols
\else % if luatex or xetex
  \usepackage{unicode-math}
  \defaultfontfeatures{Scale=MatchLowercase}
  \defaultfontfeatures[\rmfamily]{Ligatures=TeX,Scale=1}
\fi
\usepackage{lmodern}
\ifPDFTeX\else  
    % xetex/luatex font selection
    \setmainfont[]{TeX Gyre Termes}
  \setmathfont[]{TeX Gyre Termes Math}
\fi
% Use upquote if available, for straight quotes in verbatim environments
\IfFileExists{upquote.sty}{\usepackage{upquote}}{}
\IfFileExists{microtype.sty}{% use microtype if available
  \usepackage[]{microtype}
  \UseMicrotypeSet[protrusion]{basicmath} % disable protrusion for tt fonts
}{}
\usepackage{xcolor}
\usepackage[inner=3cm,outer=4cm,top=3cm,bottom=4cm,headsep=22pt,headheight=11pt,footskip=33pt,ignorehead,ignorefoot,heightrounded]{geometry}
\setlength{\emergencystretch}{3em} % prevent overfull lines
\setcounter{secnumdepth}{-\maxdimen} % remove section numbering
% Make \paragraph and \subparagraph free-standing
\makeatletter
\ifx\paragraph\undefined\else
  \let\oldparagraph\paragraph
  \renewcommand{\paragraph}{
    \@ifstar
      \xxxParagraphStar
      \xxxParagraphNoStar
  }
  \newcommand{\xxxParagraphStar}[1]{\oldparagraph*{#1}\mbox{}}
  \newcommand{\xxxParagraphNoStar}[1]{\oldparagraph{#1}\mbox{}}
\fi
\ifx\subparagraph\undefined\else
  \let\oldsubparagraph\subparagraph
  \renewcommand{\subparagraph}{
    \@ifstar
      \xxxSubParagraphStar
      \xxxSubParagraphNoStar
  }
  \newcommand{\xxxSubParagraphStar}[1]{\oldsubparagraph*{#1}\mbox{}}
  \newcommand{\xxxSubParagraphNoStar}[1]{\oldsubparagraph{#1}\mbox{}}
\fi
\makeatother


\providecommand{\tightlist}{%
  \setlength{\itemsep}{0pt}\setlength{\parskip}{0pt}}\usepackage{longtable,booktabs,array}
\usepackage{calc} % for calculating minipage widths
% Correct order of tables after \paragraph or \subparagraph
\usepackage{etoolbox}
\makeatletter
\patchcmd\longtable{\par}{\if@noskipsec\mbox{}\fi\par}{}{}
\makeatother
% Allow footnotes in longtable head/foot
\IfFileExists{footnotehyper.sty}{\usepackage{footnotehyper}}{\usepackage{footnote}}
\makesavenoteenv{longtable}
\usepackage{graphicx}
\makeatletter
\newsavebox\pandoc@box
\newcommand*\pandocbounded[1]{% scales image to fit in text height/width
  \sbox\pandoc@box{#1}%
  \Gscale@div\@tempa{\textheight}{\dimexpr\ht\pandoc@box+\dp\pandoc@box\relax}%
  \Gscale@div\@tempb{\linewidth}{\wd\pandoc@box}%
  \ifdim\@tempb\p@<\@tempa\p@\let\@tempa\@tempb\fi% select the smaller of both
  \ifdim\@tempa\p@<\p@\scalebox{\@tempa}{\usebox\pandoc@box}%
  \else\usebox{\pandoc@box}%
  \fi%
}
% Set default figure placement to htbp
\def\fps@figure{htbp}
\makeatother
% definitions for citeproc citations
\NewDocumentCommand\citeproctext{}{}
\NewDocumentCommand\citeproc{mm}{%
  \begingroup\def\citeproctext{#2}\cite{#1}\endgroup}
\makeatletter
 % allow citations to break across lines
 \let\@cite@ofmt\@firstofone
 % avoid brackets around text for \cite:
 \def\@biblabel#1{}
 \def\@cite#1#2{{#1\if@tempswa , #2\fi}}
\makeatother
\newlength{\cslhangindent}
\setlength{\cslhangindent}{1.5em}
\newlength{\csllabelwidth}
\setlength{\csllabelwidth}{3em}
\newenvironment{CSLReferences}[2] % #1 hanging-indent, #2 entry-spacing
 {\begin{list}{}{%
  \setlength{\itemindent}{0pt}
  \setlength{\leftmargin}{0pt}
  \setlength{\parsep}{0pt}
  % turn on hanging indent if param 1 is 1
  \ifodd #1
   \setlength{\leftmargin}{\cslhangindent}
   \setlength{\itemindent}{-1\cslhangindent}
  \fi
  % set entry spacing
  \setlength{\itemsep}{#2\baselineskip}}}
 {\end{list}}
\usepackage{calc}
\newcommand{\CSLBlock}[1]{\hfill\break\parbox[t]{\linewidth}{\strut\ignorespaces#1\strut}}
\newcommand{\CSLLeftMargin}[1]{\parbox[t]{\csllabelwidth}{\strut#1\strut}}
\newcommand{\CSLRightInline}[1]{\parbox[t]{\linewidth - \csllabelwidth}{\strut#1\strut}}
\newcommand{\CSLIndent}[1]{\hspace{\cslhangindent}#1}


\addtokomafont{disposition}{\rmfamily}
\setkomafont{pageheadfoot}{\normalfont\normalcolor\footnotesize}
\setkomafont{pagenumber}{\normalfont\normalcolor\footnotesize}

% ------------------
% Headers/Footers
% ------------------
\usepackage{fancyhdr}
\pagestyle{fancy}
\fancyhf{}
\fancyhead[L,C,R]{}
\fancyfoot[L,C]{}
\fancyfoot[R]{\thepage}
\renewcommand{\headrulewidth}{1pt}
\fancypagestyle{plain}{%
    \renewcommand{\headrulewidth}{0pt}%
    \fancyhf{}%
    \fancyfoot[R]{\thepage}%
}
\renewcommand\footnoterule{\rule{\linewidth}{0.1pt}\vspace{5pt}}
\usepackage{booktabs}
\usepackage{caption}
\usepackage{longtable}
\usepackage{colortbl}
\usepackage{array}
\usepackage{anyfontsize}
\usepackage{multirow}
\usepackage{wrapfig}
\usepackage{float}
\usepackage{pdflscape}
\usepackage{tabu}
\usepackage{threeparttable}
\usepackage{threeparttablex}
\usepackage[normalem]{ulem}
\usepackage{makecell}
\usepackage{xcolor}
\makeatletter
\@ifpackageloaded{float}{}{\usepackage{float}}
\floatstyle{plain}
\@ifundefined{c@chapter}{\newfloat{hypothesis}{h}{lohyp}}{\newfloat{hypothesis}{h}{lohyp}[chapter]}
\floatname{hypothesis}{H}
\newcommand*\listofhypothesiss{\listof{hypothesis}{List of Hs}}
\makeatother
\makeatletter
\@ifpackageloaded{caption}{}{\usepackage{caption}}
\AtBeginDocument{%
\ifdefined\contentsname
  \renewcommand*\contentsname{Table of contents}
\else
  \newcommand\contentsname{Table of contents}
\fi
\ifdefined\listfigurename
  \renewcommand*\listfigurename{List of Figures}
\else
  \newcommand\listfigurename{List of Figures}
\fi
\ifdefined\listtablename
  \renewcommand*\listtablename{List of Tables}
\else
  \newcommand\listtablename{List of Tables}
\fi
\ifdefined\figurename
  \renewcommand*\figurename{Figure}
\else
  \newcommand\figurename{Figure}
\fi
\ifdefined\tablename
  \renewcommand*\tablename{Table}
\else
  \newcommand\tablename{Table}
\fi
}
\@ifpackageloaded{float}{}{\usepackage{float}}
\floatstyle{ruled}
\@ifundefined{c@chapter}{\newfloat{codelisting}{h}{lop}}{\newfloat{codelisting}{h}{lop}[chapter]}
\floatname{codelisting}{Listing}
\newcommand*\listoflistings{\listof{codelisting}{List of Listings}}
\makeatother
\makeatletter
\makeatother
\makeatletter
\@ifpackageloaded{caption}{}{\usepackage{caption}}
\@ifpackageloaded{subcaption}{}{\usepackage{subcaption}}
\makeatother

\usepackage{bookmark}

\IfFileExists{xurl.sty}{\usepackage{xurl}}{} % add URL line breaks if available
\urlstyle{same} % disable monospaced font for URLs
\hypersetup{
  pdftitle={title},
  colorlinks=true,
  linkcolor={blue},
  filecolor={Maroon},
  citecolor={RoyalBlue},
  urlcolor={Blue},
  pdfcreator={LaTeX via pandoc}}


\title{title}
\usepackage{etoolbox}
\makeatletter
\providecommand{\subtitle}[1]{% add subtitle to \maketitle
  \apptocmd{\@title}{\par {\large #1 \par}}{}{}
}
\makeatother
\subtitle{subtitle}

\author{
{\bfseries \normalsize Isaiah Espinoza}%
 \\%
 \small University of Maryland, Department of Government and
Politics \\%
{\footnotesize \url{gespinoz@umd.edu}} \\\vspace{10pt}
}

\date{2024-10-16}

\begin{document}

% for some reason this does not work in header
\renewcommand{\abstractname}{Abstract.}

% add the short title to the fancy header
\fancyhead[R]{A Short title}
\fancyhead[L]{Isaiah}

\maketitle
%\noindent \rule{\linewidth}{.5pt}
\newcommand{\datemodified}[1]{Last modified: #1}
{\centering
\hbox{}
{\datemodified{2024-11-20} \par}}
%\noindent \rule{\linewidth}{.5pt}


\setstretch{2}
\pagebreak

\subsection{Introduction}\label{introduction}

Election administration officials make efforts to sustain public trust
and confidence in the fairness and accuracy of elections, and attempt to
boost such confidence where ever it may be deprived. Concerns for safety
have developed among election staff and voters in more recent elections.
Regular measures are taken to enhance the
\emph{trustworthiness}\footnote{I adopt a distinction made between
  \emph{trustworthiness} and \emph{trust} in elections
  (\citeproc{ref-stewart2022}{Stewart 2022}). The ``\emph{worthiness}''
  of one's trust in the conduct and administration of elections is based
  on the extent that outcomes of an election reasonably follow the rules
  prescribed and can be adjudicated as such. ``If the process is
  conducted fairly and competently, and the results are determined by
  the actions of voters\ldots we can call this a trustworthy election''
  (\citeproc{ref-stewart2022}{Stewart 2022, 237}).
  \emph{Trustworthiness} is distinguished from \emph{trust} as a
  psychological construct, ``\ldots the conclusion reached by the public
  about the functioning of the process''
  (\citeproc{ref-stewart2022}{Stewart 2022, 237}). To put in other
  terms, trustworthiness is built by the structure, procedures, and
  practices of the institution, in this case election administration.
  The public's trust in elections, however, is amendable to an
  indefinite number of factors that may be seemingly unrelated to the
  formal procedures of election administration. To illustrate, my car is
  trustworthy because I've known it to function quite well as a car
  should; it passes whatever criteria upon inspection, its sufficiently
  fueled, and is in generally good working order. I have every reason to
  deem it worthy of my trust. However, I don't trust my car because I am
  sure it is haunted.} of the electoral process through practices meant
to improve the conduct, transparency, or overall administration of
elections in the United States.

Although election officials undertake great efforts to enhance the
trustworthiness of election administration, public \emph{trust} in
elections is a psychological construct influenced by many things outside
of election official control such as partisanship or elite rhetoric
(\citeproc{ref-hooghe2018}{Hooghe 2018};
\citeproc{ref-sances2015}{Sances and Stewart 2015}). Moreover, a
person's evaluation of the election in hindsight is often influenced by
the election outcome itself (\citeproc{ref-daniller2019}{Daniller and
Mutz 2019}; \citeproc{ref-stewart2022}{Stewart 2022}). Thus, measures
taken by election officials can be undermined, trivialized, or made
irrelevant depending on how one feels after the election results have
come out.

Such volatile attitudes and evaluations post-election can leave a
lasting impression that election officials must contend with upon the
next election cycle (\citeproc{ref-bowler2024}{Bowler and Donovan 2024};
\citeproc{ref-levendusky2024}{Levendusky et al. 2024}). For instance, we
have witnessed many people's outright refusal to accept the 2020 U.S.
election results as legitimate despite consistent review of the evidence
confirming the results as fair and accurate. Such a case demonstrates
that public trust in elections is, at best, only partial to
trustworthiness of election administration in the United
States\footnote{It is important to point out here that election
  administration is just one institution that interacts with many other
  institutions and processes which form the electoral system \emph{writ
  large} (\citeproc{ref-stewart2022}{Stewart 2022}). The way in which I
  discuss the institution of election administration, as well as the way
  in which I discuss and distinguish public trust in elections from
  broader conceptions of public trust, is borrowed substantively from
  Stewart (\citeproc{ref-stewart2022}{2022})'s conceptual framework of
  the same. Distinctions between \emph{trust} and \emph{trustworthiness}
  and the notion of election administration as an institution are
  already apparent inspirations convenient for situating the theory of
  this research inquiry.}. Or in other words, there is only so much that
election officials can do to sustain or improve public trust in election
administration.

Even though election officials can do a lot to secure election
integrity, especially under intense scrutiny, there's not much they can
do to cement public confidence after election night passes. At best,
election officials can ease public anxieties prior to election night. As
mentioned, this usually consists of enhancing procedures and practices
and adapting to advancing election technology.

One point of contention that election officials have faced in the past
regard evaluation of election workers. Previous literature has focused
on how voter interaction with election workers
(\citeproc{ref-claassen2008}{Claassen et al. 2008}), or the voter
experience generally (\citeproc{ref-atkeson2007}{Atkeson and Saunders
2007}), influences evaluations of election administration. As such,
election worker competency has been examined as a factor significant to
evaluations of performance of elections (\citeproc{ref-hall2007}{T.
Hall, Monson, and Patterson 2007}; \citeproc{ref-hall2009}{T. E. Hall,
Quin Monson, and Patterson 2009}). However, considering that individual
perceptions and preconceived notions play a huge role in cognition
(\citeproc{ref-cikara2014}{Cikara and Bavel 2014};
\citeproc{ref-vanbavel2021}{Van Bavel and Packer 2021}), it is
reasonable to expect that the group an election worker hails from would
be an important influence upon the voter's perception or evaluation of
their own experience as well as subsequent evaluations of the electoral
process. Supposing such is the case, we can expect that information
about \emph{who} (i.e., which groups) election officials are targeting
in publicized recruitment efforts would lessen particular election
anxieties. That is to say, it is reasonable to expect that telling
people \emph{who} will be working as election staff and volunteers would
ease election anxieties, therefore improve confidence that the election
will be conducted fairly, accurately, and safe for all involved.

In this paper, I report results from a recent survey experiment
administered to test whether publicized efforts to recruit veterans to
work as election staff and volunteers would ease election anxieties
(e.g., concern for electoral fairness and accuracy, as well as concerns
for voter safety). Results of the survey experiment support the
hypothesis that emphasizing veterans as the target of election worker
recruitment efforts eases pre-elections anxieties to an extent. In
particular, among those who read the treatment vignette, confidence that
the elections would be fair and accurate was higher in comparison. In
addition, expectations of electoral fraud were lower, as were concerns
for voter safety compared to those who read a control vignette. Notable
is that there was a significant increase in confidence, and lower
expectations of fraud, among those who believe that the 2020 election
was illegitimate.

\begin{itemize}
\tightlist
\item
  when distinguished by partisanship\ldots{}
\item
  moreover, broken down by other demographic variables such as race,
  gender, education, etc.
\item
  holding these factors constant demonstrates a
  {[}increased/decreased{]} likelihood that a respondent's confidence
  would likely {[}increase/decrease{]}, expectations of fraud would
  likely {[}increase/decrease{]}, and concerns for safety would lessen
  among those in the treatment group compared to those in the control.
  These results allow us to attribute the differences to the treatment
  stimulus, namely, the prospect of having veterans serve as election
  staff or volunteers.
\end{itemize}

\subsection{Background: Election Administration and Public
Confidence}\label{background-election-administration-and-public-confidence}

Election officials have tried hard to inspire confidence in the
administration and conduct of elections by improving the degree to which
elections are trustworthy. Development and implementation of procedures
such as post-election auditing of ballots and logic-and-accuracy testing
of ballot tabulation equipment are prominent examples adding to the long
history of efforts to enhance the trustworthiness of election
administration in the United States.

Prior to the year 2000, one of the main issues facing election
administration was recruiting enough election workers to volunteer at
the polls (i.e., poll workers) (\citeproc{ref-maidenberg1996}{Maidenberg
1996}). Election worker recruitment is still much of an issue in the
current era as it was then, perhaps worse
(\citeproc{ref-ferrer2024}{Ferrer, Thompson, and Orey 2024}). In
addition to ensuring election admin offices were adequately staffed, the
controversy of the 2000 general election made the public more attentive
to issues concerning the conduct and administration of elections. In
particular, voting technology (\citeproc{ref-herrnson2009}{Herrnson,
Niemi, and Hanmer 2009}) and election worker competence was a of
interest in election studies (\citeproc{ref-claassen2008}{Claassen et
al. 2008}; \citeproc{ref-hall2007}{T. Hall, Monson, and Patterson 2007};
\citeproc{ref-hall2009}{T. E. Hall, Quin Monson, and Patterson 2009}).
Following the passage of the Help America Vote Act in 2002, election
officials efforts to boost public confidence in the conduct and
administration of elections revolved primarily around the accuracy of
vote counts, ballot tabulation equipment or voting machines, the
commitment of election staff, and more
(\citeproc{ref-atkeson2007}{Atkeson and Saunders 2007}). These somewhat
generic issues of election worker staffing, competence, and voting
technology were likely mere stepping stones leading to more intense
election administration issues.

In 2024, election officials made valiant efforts to boost public
confidence in the fairness and accuracy of elections within an
intensified political climate that appeared quite hostile to election
officials (\citeproc{ref-brennancenterforjustice2024}{Brennan Center for
Justice 2024}; \citeproc{ref-edlin2024}{Edlin and Norden 2024}).
Although polling around the time indicated that most people thought that
U.S. elections would be run at least somewhat well
(\citeproc{ref-nadeem2024}{Nadeem 2024}), many election officials
nationwide took efforts to assuage the worry of those most skeptical.
Enhancing trustworthiness in election administration is a regular duty
for election officials, but baseless allegations regarding the fairness
and accuracy of the 2020 election loomed ever larger, and the fear of
political violence became prominent for both election officials and the
public. Safety concerns for election staff, volunteers, and voters alike
only added to, and perhaps exacerbated, public concern for the integrity
of election administration in the United States. Especially as the
steady increase in election official turnover seems to be increasing
even more (\citeproc{ref-ferrer2024}{Ferrer, Thompson, and Orey 2024}).

Election anxiety was high in the lead up to the 2024 elections in the
United States. Concerns for voter safety and the prospect of political
violence remained prescient and compelled many local officials to
prepare for the worst (\citeproc{ref-doubek2024}{Doubek 2024};
\citeproc{ref-edlin2024}{Edlin and Norden 2024}). Election officials in
Washoe County, Nevada, installed panic buttons for election staff that
would alert a monitoring center to summon law enforcement
(\citeproc{ref-lincoln2024}{Lincoln 2024}). Nevada also passed a law
making it a felony to harass, threaten, or intimidate election workers
(\citeproc{ref-nevadasecretaryofstate2023}{Nevada Secretary of State
2023}). Leading up to election day, news outlets reported that election
work had become a seemingly dangerous job (\citeproc{ref-wire2024}{Wire
et al. 2024}). A Brennan Center survey report stated that,
``\ldots large numbers of election officials report having experienced
threats, abuse, or harassment for doing their jobs''
(\citeproc{ref-edlin2024}{Edlin and Norden 2024}). Concerns over the
fairness of elections and accuracy of vote counts intensified,
heightening concerns over the prospect of political violence and, in
turn, engendered increased worry for the safety of voters and election
workers alike.

Suffice to say, pre-election anxiety consists of more than concerns over
fairness and accuracy of vote counts in light of added safety concerns.
It is not hard to recognize that increased tension in the pre-election
period makes for a volatile political environment.

\subsection{Literature Reivew}\label{literature-reivew}

Assessing the public's \emph{trust} in elections has not been
straightforward. Inquiry into public trust in elections has been
approached by scholars of political science in many different ways
(\citeproc{ref-cook2005}{Cook and Gronke 2005}), often distinguishable
by the scope of the research question and more or less constrained by
the particular conception of public trust. Quite often, trust in
election administration is conflated with trust in government writ
large, government legitimacy, government or system responsiveness, or
even satisfaction with democracy (\citeproc{ref-daniller2019}{Daniller
and Mutz 2019}). At this level, not only is the level of public trust in
elections sometimes vague, but there's little consideration over the
difference between such attitudes pre-election and post-election. In
contrast, a considerable amount of research tends to conceive of public
trust in accordance with the institution in question
(\citeproc{ref-atkeson2007}{Atkeson and Saunders 2007};
\citeproc{ref-hooghe2018}{Hooghe 2018}).

Election administration is just one part of the larger set of
institutions which form the electoral system. As such, the performance
of the institution along with the rest ``\ldots lends credibility to the
outcome of an election: whether it is considered by citizens and the
international community to be fair and legitimate.''
(\citeproc{ref-stewart2022}{Stewart 2022, 236}).

Trust and confidence in the conduct of elections concerns aspects of
elections that fall squarely within the institution of election
administration. At this level, for instance, public trust is ascertained
by capturing assessments about the perceived accuracy of vote counts
(e.g., whether votes are/were counted as intended). Or in other words,
public trust is ascertained by the extent the public perceives the
institution of election administration as trustworthy.

Intuitively, enhancing public trust in elections would best be
accomplished by enhancing the \emph{trustworthiness} of the institution,
i.e., consistently doing the things that election officials already
regularly do come election time. However, trust and confidence in
elections has become ever more precarious over the last few election
cycles. Especially considering public polling data since 2000 shows that
confidence that votes were, or would be, counted as intended was in a
consistent decline despite efforts towards bolstering election integrity
and trustworthiness (\citeproc{ref-sances2015}{Sances and Stewart
2015}). This is even more pronounced considering the role that
partisanship has had on such confidence over accuracy of vote count
(\citeproc{ref-sances2015}{Sances and Stewart 2015};
\citeproc{ref-stewart2022}{Stewart 2022}).

There's also stark difference in public trust before the election has
occurred compared to after, a phenomenon referred to as the
``winner-loser gap''; the ``winners'' are those who supported the
winning candidate and the ``losers'' are those who supported the losing
candidate. Much research has been dedicated to analyzing the sentiment
of electoral winners vs losers, and vice versa
(\citeproc{ref-daniller2019}{Daniller and Mutz 2019};
\citeproc{ref-nadeau1993}{Nadeau and Blais 1993}). Opinions of electoral
trust gathered after the election has occurred are limited considering
the well-recognized impact that the electoral outcome itself has on
feelings of public trust in elections
(\citeproc{ref-daniller2019}{Daniller and Mutz 2019}).

As such, it is questionable whether we can characterize public trust in
the pre-election period as the same trust after the election results
have come out\footnote{There's little that could justify a conceptual
  distinction between public trust pre-election and public trust
  post-election. The temporal element renders the difference between
  pre- and post-election operative rather than conceptual.}. The former
is \emph{anticipatory}---i.e., the kind that is more or less anxious
given the uncertainties surrounding the election. The latter is
\emph{evaluative}---a judgement discerned empirically in hindsight after
the election event has occurred. The evaluative form of trust lends to
inquiries interested in public opinion on regime legitimacy,
satisfaction with democracy, and perceived responsiveness
(\citeproc{ref-daniller2019}{Daniller and Mutz 2019}). This study
focuses primarily on that \emph{anticipatory} kind of trust and
confidence, which speaks more to electoral anxieties over the
institution of election administration. Not to mention that by
definition, one feels anxiety over impending uncertainties located in
the future\footnote{Note that \emph{anxiety} is taken to be the inverse
  of \emph{trust and confidence}, as in, one who is less confident in
  this or that is thereby more anxious over the matter in question. As
  such, anxiety may equivalently be discussed as trust/confidence
  without issue except for framing of said discussion. Electoral anxiety
  is used in place of electoral trust/confidence simply to emphasize the
  anticipatory nature of the attitudes in question.}.

The primary interest of this study concerns \emph{election anxieties},
which is distinguished into two categories: fairness and accuracy, and
voter safety. Of course, the direction of electoral outcomes are a
particular source of electoral anxiety, but only the election outcome
itself can resolve any nail-biting stress over its anticipated
direction. In contrast, the anxieties of interest here are able to be
relaxed to some extent prior to the election given some intervention.

Interventions aimed at improving public trust, or rather, aimed at
easing election anxieties may not be within the usual repertoire
election officials draw upon. Regardless of the measures taken by
election officials to boost public confidence in the
\emph{trustworthiness} of election administration (e.g., conducting
audits, testing election machines), public \emph{trust} and confidence
in elections more generally is apt to shift dramatically post-election
based on factors such as partisanship, elite rhetoric, particular state
policies, and more (\citeproc{ref-carter2024}{Carter et al. 2024};
\citeproc{ref-coll2024a}{Coll and Clark 2024};
\citeproc{ref-nadeau1993}{Nadeau and Blais 1993}). Moreover, prior
research has found that evaluation of election workers themselves are an
important factor when it comes to levels of public confidence in the
electoral process (\citeproc{ref-claassen2008}{Claassen et al. 2008};
\citeproc{ref-hall2007}{T. Hall, Monson, and Patterson 2007};
\citeproc{ref-hall2009}{T. E. Hall, Quin Monson, and Patterson 2009}).
Such studies focused on the quality of the voter experience with
reference to the interaction between voter and election worker.

Beyond the general competence of election workers, however, the quality
of the voter experience may be influenced merely by impressions about
\emph{who} comprises election staff and volunteers. Political and other
social science researchers have recognized for some time the power that
group identity can have over attitudes and perception
(\citeproc{ref-vanbavel2021}{Van Bavel and Packer 2021};
\citeproc{ref-xiao2016}{Xiao, Coppin, and Van Bavel 2016};
\citeproc{ref-xiao2012}{Xiao and Van Bavel 2012}). As such, we can
expect that information identifying the particular groups being
recruited to serve as election staff will be enough to improve trust and
confidence in election administration, i.e., ease election anxieties. It
was hypothesized that information on recruitment efforts targeting
military veterans to work as election staff would ease election
anxieties. The next section elaborates further on why military veterans
are of particular interest.

\subsection{Why Veterans}\label{why-veterans}

Election officials are likely to be agnostic as to who dedicates their
time to civil service such as election work. Staffing issues have been
an issue since at least the 1990s (\citeproc{ref-ferrer2024}{Ferrer,
Thompson, and Orey 2024}; \citeproc{ref-maidenberg1996}{Maidenberg
1996}). In 2020, such efforts were made far more difficult by the
COVID-19 pandemic (\citeproc{ref-abbate2020a}{Abbate 2020};
\citeproc{ref-mena2020}{Mena 2020}). Sure, it makes sense to recruit
veterans, but no more than any other group. After all, ensuring election
offices are adequately staffed is everyone's problem. So there's no
special reason to target veterans for recruitment above other groups.
Indeed, there's no reason to discriminate recruitment efforts at all if
the point is purely to fill staffing vacancies. However the interest in
veterans as a special group to consider arose in light of increased and
sudden efforts to recruit veterans into election work.

The reason why veterans arose as a special subset of the population to
consider for this inquiry is because there was a sudden push to target
veterans for recruitment efforts that arose shortly after the events on
Capitol Hill on January 6th, 2021. After the 2020 election, large
efforts were made to recruit military veterans and their families to
work or volunteer as election staff
(\citeproc{ref-nflfootballoperations2022}{NFL Football Operations 2022};
\citeproc{ref-wetheveterans2022}{We The Veterans 2022}). Prior to that
point, young people were sometimes given special mention as targets of
election worker recruitment efforts\footnote{The cited study defined
  ``insurrectionist sentiments'' as, ``\ldots willingness to support
  violent efforts to overturn the results of an election in favor of
  another, unelected political leader.'' (\citeproc{ref-pape2024}{Pape
  et al. 2024, 7}). It should be noted, however, that despite the
  general definition, the operative items used to measure such
  sentiments ask about Donald Trump specifically, not an ``unelected
  political leader''.} (\citeproc{ref-herndon2020}{Herndon 2020};
\citeproc{ref-powerthepolls2020}{Power the Polls 2020}). Generally,
however, recruitment efforts cast a wide net, indiscriminate of who
applies (\citeproc{ref-conde2020}{Conde 2020};
\citeproc{ref-ross2020}{Ross 2020}).

One can speculate that the motivation to associate military veterans
with civic engagement and democracy may be intended to counter negative
perceptions and impressions given by the proportion of veteran service
members arrested for taking part in the events on January 6th
(\citeproc{ref-jensen2022a}{Jensen, Yates, and Kane 2022};
\citeproc{ref-loewenson2023}{Loewenson 2023};
\citeproc{ref-milton2021}{Milton and Mines 2021}). Especially with
regard to research demonstrating that willingness to support violent
efforts to overturn election results\footnote{Prior to 2021, I found one
  online recruitment brochure intent on recruiting veterans to serve as
  election workers (\citeproc{ref-studentveteransofamerica2020}{Student
  Veterans of America 2020}).} (in support of Trump) is, on average,
more common among veterans than among matched samples of non-veterans
(\citeproc{ref-pape2024}{Pape et al. 2024}). This is in addition to a
strengthened association portrayed in media outlets between military
veterans and militias (\citeproc{ref-steinhauer2020}{Steinhauer 2020}).
Although such media output reeks of the contemporary political context,
prior research has substantiated such a connection between veterans and
militia groups. A. Cooter (\citeproc{ref-cooter2024}{2024}) notes from
her 3-year ethnographic fieldwork among Michigan militia members that,
``\ldots approximately 40\% of militia leaders and 30\% of members had
previous military experience. Most of these veterans actively sought out
such groups, as opposed to being recruited by them''
(\citeproc{ref-cooter2024}{2024}; see also \citeproc{ref-cooter2013}{A.
B. Cooter 2013}). Thus, countering such associations by promoting a
different image of veterans to the mass public and veterans alike seems
like a reasonable motivation. Yet such speculation is just that.

That being said, the general public perception, attitudes, or even
stereotypes about military veterans are significant to consider.
Although military recruitment shows a downward trend as of late, public
perception of veterans are overwhelmingly positive
(\citeproc{ref-kleykamp2023}{Kleykamp, Schwam, and Wenig 2023}).
Veterans are a particularly potent group where mention of veteran status
seems to have a positive, calming, or nullifying effect on attitudes.
For instance, recent research shows that, during his campaign in the
2020 Democratic Primaries, Pete Buttigieg's military background
mitigated discrimination against him when he was presented as a veteran
married to man (\citeproc{ref-magni2024}{Magni and Reynolds 2024}).
Similar research has also found that a candidate's veteran status
affords them better evaluations regarding competency in particular issue
areas (e.g., war competence) (\citeproc{ref-hardy2019}{Hardy et al.
2019}). Moreover, veteran status seems to mitigate or nullify usual
stigmas associated with mental illness. That is to say, there is
negative stigma associated with mental illness
(\citeproc{ref-corrigan2002}{Corrigan et al. 2002}) and such stigma
incurs labor market discrimination (\citeproc{ref-hipes2016}{Hipes et
al. 2016}), but evidence suggests that veteran status overrides such
stigma and discrimination (\citeproc{ref-maclean2014}{MacLean and
Kleykamp 2014}). Or, in another light, mental illness is seemingly more
\emph{understandable} (i.e., permissible) for veterans given the
presumptive reasons for their mental strife. And media framing as such
plays an important role on public perception
(\citeproc{ref-kleykamp2015}{Kleykamp and Hipes 2015}).

\subsection{Research Design and
Method}\label{research-design-and-method}

To test the theory that publicized efforts to recruit veterans to work
as election staff and volunteers would ease election anxieties, a recent
survey experiment was embedded in a survey developed and conducted by
the Center for Democracy and Civic Engagement (CDCE) at the University
of Maryland (UMD). The survey was fielded from August 29th, 2024 to
September 18th, 2024 on a non-probability sample of 1,287 U.S. citizens
18 years of age or older.

The primary independent variable is the prospect of military veterans as
poll workers. It was considered that veterans would elicit high degrees
of approval and support for whatever cause or issue presented. For
example, I'd expect high approval and support for recruiting veterans to
baseball teams just as much as for recruiting them to work as election
staff. Asking participants whether they supported programs intent on
recruiting veterans to work as election staff would garner support
regardless of the target group. Considering that support and admiration
for veterans is generally high among the population
(\citeproc{ref-kleykamp2023}{Kleykamp, Schwam, and Wenig 2023}), then
discerning the impact on one's confidence would require a survey
experiment and questions designed to determine whether veterans have any
special effect on electoral confidence and voter safety distinct from
the generally high admiration observed most often.

Survey participants were randomly assigned to read one of two fabricated
news articles (i.e., vignettes) about recruitment efforts for election
jobs. Participants read either a treatment or control vignette, which
was a fabricated news article about efforts in Maricopa County, AZ to
recruit election staff and volunteers for the 2024 general election. The
treatment vignette referred to a program designed to recruit veterans
and their family members and describes an interviewee ``Jordan Braxton''
as an Army veteran; whereas the control vignette simply omitted any
mention of veterans and their family members, and didn't describe
``Jordan Braxton'' as an Army veteran. Beyond those small differences
and the headlines, the article vignettes are identical. Any effects can
be attributed to the information about veterans in the treatment
vignette\footnote{Complete text of treatment and control vignettes are
  included in the Appendix}.

It should be noted that Maricopa County, AZ was chosen as the setting of
the story in the vignettes due to the increased scrutiny levied toward
election administration there after the 2020 election
(\citeproc{ref-giles2021}{Giles 2021};
\citeproc{ref-maricopacountyelectionsdepartment2022}{Maricopa County
Elections Department 2022}). In addition, the survey includes a series
of items which ask specifically about Maricopa County, AZ and a
comparable series of questions about one's local area\footnote{The
  setting of the vignettes and inclusion of duplicate questions in the
  survey are highlighted and discussed later.}.

Over the past two decades it became commonplace for national polls to
gauge public confidence in election administration by asking some
variety of the question, ``How confident are you that your vote {[}will
be/was{]} counted as you intended in the most recent election?''
(\citeproc{ref-hall2009}{T. E. Hall, Quin Monson, and Patterson 2009};
\citeproc{ref-sances2015}{Sances and Stewart 2015};
\citeproc{ref-stewart2022}{Stewart 2022}). In addition, since 2008, the
Survey of the Performance of American Elections (SPAE) has included a
good number of relevant questions to more thoroughly assess trust and
confidence in election administration. Such questions inquire into the
voter experience with the institution more directly. This study borrows,
modifies, or takes inspiration from certain questions found within the
2022 SPAE and other survey items used by the Pew Research Center
(\citeproc{ref-dunn2018}{Dunn 2018}). These survey items comprise the
dependent variables\footnote{All survey items are included in the
  Appendix.}.

The outcomes of interest (i.e., dependent variables) are broadly
referred to as \emph{election anxieties}, but more specifically refer to
confidence over fairness and accuracy of the election administration;
expectations of fraudulent activity intent on manipulating the electoral
outcomes; and concern for voter safety, i.e., comfort voting in-person
as it relates to expectations of violence, threats, or intimidation. The
following presents multiple one-sided directional hypotheses. Previous
research on public perception and regard for U.S. military veterans
gives credence to the directional hypotheses
(\citeproc{ref-kleykamp2018}{Kleykamp, Hipes, and MacLean 2018};
\citeproc{ref-kleykamp2023}{Kleykamp, Schwam, and Wenig 2023}).

The first outcome of interest regards \emph{trust and confidence} in
election administration, which includes the integrity and security of
the electoral process, administration, technology, and people or
organizations to fairly and accurately conduct elections. Survey items
measured the degree of confidence that votes will be counted as voters
intend, confidence that electoral systems are secure from technological
threats, perceived commitment of election staff, confidence that
outcomes will be fair, as well as confidence that the voting process
will be fair.

The first hypothesis is as follows:

\(H_{1}:\) Trust and confidence in election administration will be
higher among those who were presented information about efforts to
recruit veterans as election workers compared to those who were not

In total, six survey items measure different aspects of trust and
confidence in election administration. These items were analyzed
individually but were also transformed into a composite scale of trust
and confidence.

The literature concerning public regard for veterans gives substantial
reason to expect that publicized efforts to recruit veterans to work as
election staff and volunteers would have the effect of easing election
anxieties. As such, survey participants also responded to a series of
questions that inquired into expectations of electoral fraud---i.e.,
fraudulent activity intent on manipulating the electoral outcome.

Note, however, that an \emph{expectation} of electoral fraud implies a
degree of distrust more so than an agitated state of anxiety over the
\emph{potential} for electoral fraud. Conceptually, this may be
considered a particular kind of `anxiety' given that is it pre-election
(hence, an expectation), however, not necessarily. A person may be
rather confident with regard to how much they distrust the integrity of
electoral administration, resulting in such expectations.

For instance, the first item of the 5-item series asks participants to
report their expectation that ``There will be voter fraud, that is,
people who are not eligible to vote will vote or vote more than once''
on a 4-option response scale ranging from ``Not likely at all'' to
``very likely''. This statement, as well as the rest in the series,
asserts an expected compromise of election integrity, which speaks to
the trustworthiness of the institution. What the individual expects,
more or less, is that electoral fraud will occur, implying that
fraudulent attempts will be successful.

It was expected that the experimental stimulus would result in lower
expectation for electoral fraud among those who read the treatment
vignette compared to those who read the control vignette. The second
hypothesis is,

\(H_{2}:\) Expectations of electoral fraud will be lower among those who
were presented information about efforts to recruit veterans as election
workers compared to those who were not

Another outcome of interest regards concerns about voting in-person at a
polling site, i.e., voter safety. That is, the extent people feel that
voters are safe enough to cast a vote at polling sites free from
violence, threats of violence, or intimidation. Two survey items use
slightly different questions to inquire about the same essential
issue---i.e., perceptions of physical safety at the polls. The third
hypothesis follows,

\(H_{3}:\) Concerns for voter safety be lower among those who are
presented information about efforts to recruit veterans as election
workers compared to those who are not

Another outcome of interest concerned how particular actions of election
officials, or circumstances at polling sites, would impact the
confidence of election integrity as well as voter safety \emph{from the
perspective of survey respondents}. Survey participants responded to a
series of six statements, all prefaced with the same question,
``Regardless of whether any of these are actually the case, how would
the following impact your confidence in the fairness and accuracy of
elections conducted this November?'' The six statements were,

\begin{enumerate}
\def\labelenumi{\arabic{enumi}.}
\tightlist
\item
  Election officials test every machine used in the election to ensure
  they are secure.
\item
  Election officials conduct audits of ballots after every election to
  confirm the results were accurate.
\item
  Poll watchers affiliated with the political parties or candidates
  observe the election.
\item
  Election staff and volunteers include military veterans and their
  family members from the community.
\item
  Election staff and volunteers include lawyers from the community.
\item
  Election staff and volunteers include college students from the
  community.
\end{enumerate}

For each statement, survey participants responded by selecting one of
five response options:

\begin{enumerate}
\def\labelenumi{\arabic{enumi}.}
\tightlist
\item
  Decrease confidence a lot
\item
  Decrease confidence somewhat
\item
  No impact on confidence
\item
  Increase confidence somewhat
\item
  Increase confidence a lot
\end{enumerate}

It was expected that respondents would estimate stronger impact on their
confidence given they were assigned to read the treatment vignette
compared to those who were not. Thus, the hypothesis is,

\(H_{4}:\) Respondents who read the treatment vignette will report
stronger impact on their personal confidence in fairness and accuracy of
election administration when veterans are stated to be included among
election staff and volunteers compared to those who read the control
vignette.

The extent people believe veterans are more trustworthy as election
workers than non-veterans in general is difficult without comparable
groups to consider. Recruitment efforts in 2022 targeted young students,
lawyers, and military veterans (\citeproc{ref-wang2022}{Wang 2022}).
Thus, lawyers and college students were selected as two groups for which
responses could be compared between different statement versions.

The goal was to determine whether there is anything special in having
veterans volunteer as election staff (i.e., poll workers) compared to
other groups. That is, does it make any difference in terms of concerns
over election security, fairness, and voter safety to include veterans
as election staff and volunteers compared to other groups? The fourth
hypothesis, irrespective of the experimental stimulus, reads,

\(H_{5}:\) Respondents who read the treatment vignette will report
stronger impact on their personal confidence in fairness and accuracy of
election administration when veterans are stated to be included among
election staff and volunteers compared to identical statements about
lawyers or college students.

\subsection{Results}\label{results}

\pagebreak

\subsection{Conclusion}\label{conclusion}

\subsubsection{Limitations}\label{limitations}

\begin{itemize}
\tightlist
\item
  limitations: interpretations of the treatment effect are limited.
\item
  The sample may not be generalizable to the population.
\item
  It's hard to say whether merely mentioning veterans is enough, as
  compared to explicitly naming veterans as the target of particular
  recruitment efforts.
\item
  Interpretation is also limited considering that no other particular
  group, or groups are compared directly against the veteran treatment
  vignette, i.e., additional vignettes for other comparable groups.
\item
  Moreover, results are limited by the survey questions, questionnaire
  design, and experimental stimulus (i.e., vignette) for two reasons. -
  First, the vignettes and many survey questions ask about Maricopa
  County, AZ specifically. Adding a specific county in the vignette adds
  in a factor that cannot be accounted for without additional treatments
  that eliminate the setting as a potential influence. Moreover, the
  specificity of the setting adds in even more unexplained error---some
  people may have attitudes about the county in question, others won't,
  while others may be muffed to consider a random county in the U.S.
  they've know nothing about. Adding the county undermines the
  confidence that treatment effects are solely attributable to veterans
  to an unknown degree. - Second, questions ask about one's own local
  community in addition to identical questions which asked about
  Maricopa County, AZ. Consequently, many questions within the
  questionnaire were duplicates with that one differences. This isn't
  uncommon in surveys, however in this case, it lengthened the survey to
  a degree that likely resulted in a higher drop off rate. More
  importantly, the quality of responses were likely diminished to some
  unknown extent. Although it is possible to compare questions asked
  about Maricopa County, AZ to questions about one's local area, the
  quality of that comparison is limited by the unknown extent of fatigue
  induced by answering the same questions twice. It wasn't just that
  some questions were asked multiple times, almost all of the questions
  were duplicated; after completing a long series on Maricopa County,
  participants then answered the same questions about the local
  community. The rate to which participants dropped from the survey is
  \_\_\_, which suggests fatigue as an important factor. Survey fatigue
  is a known issue that should be taken into consideration. Comparison
  of treatment effects between the MC series and the local area series
  is undermined by the unknown influence that fatigue would have on
  response choices. Any differences couldn't confidently be attributed
  to ``my area'' vs Maricopa county. We also can't determine whether
  treatment effects are sustained when comparing the Maricopa County
  series to the local area series asked subsequently. We would have to
  assume the setting as irrelevant, which we can't reasonably do.
\end{itemize}

\pagebreak

\subsection{Appendix A: Survey Experiment
Vignettes}\label{sec-appendix-a}

\begin{figure}

\begin{minipage}{0.50\linewidth}

\subsubsection{Treatment Vignette}\label{treatment-vignette-1}

\textbf{Local Military Veterans Recruited for Election Jobs in Maricopa
County}\\
\strut \\
\strut ~~PHOENIX (AP) --- Election officials in Maricopa County,
Arizona, announced a program designed to recruit military veterans and
their family members from the community to serve as election
administrators, including election polling place workers, temporary
workers, and full-time staff. As the U.S. general elections in November
near, election officials must fill several thousand temporary positions
and hundreds of other open positions to ensure sufficient staffing for
the 2024 elections and beyond.\\
\strut \\
\strut ~~Army veteran Jordan Braxton just joined the elections
workforce. Jordan believes their role is important to ensuring a secure,
accurate, and transparent election, ``Many places are short on staff
this election cycle. I served my country in the Army, and I want to do
my part as a veteran and a citizen to ensure that everyone trusts the
process and the outcome of the election.''

\end{minipage}%
%
\begin{minipage}{0.50\linewidth}

\subsubsection{Control Vignette}\label{control-vignette-1}

\textbf{Local Residents Recruited for Election Jobs in Maricopa
County}\\
\strut \\
\strut ~~PHOENIX (AP) ---Election officials in Maricopa County, Arizona,
announced a program to recruit members of the community to serve as
election administrators, including election polling place workers,
temporary workers, and full-time staff. As the U.S. general elections in
November near, election officials must fill several thousand temporary
positions and hundreds of other open positions to ensure sufficient
staffing for the 2024 elections and beyond.\\
\strut \\
\strut ~~Jordan Braxton just joined the elections workforce. Jordan
believes their role is important to ensuring a secure, accurate, and
transparent election, ``Many places are short on staff this election
cycle. I want to do my part as a citizen to ensure that everyone trusts
the process and the outcome of the election.''

Text of the Control Condition: Recruitment of Community
Members\end{minipage}%

\end{figure}%

\subsection{Appendix B: Sample Demographics}\label{sec-appendix-b}

There should be nothing below here

\begin{table}

\caption{\label{tbl-desc-pdf}Description of Sample Demographics}

\centering{

\centering\begingroup\fontsize{8}{10}\selectfont

\resizebox{\ifdim\width>\linewidth\linewidth\else\width\fi}{!}{
\begin{tabular}[t]{>{\raggedright\arraybackslash}p{20em}rr}
\toprule
Name & n & Pct\\
\midrule
\addlinespace[0.3em]
\multicolumn{3}{l}{\textbf{Experiment Condition}}\\
\cellcolor{gray!10}{\hspace{1em}Control} & \cellcolor{gray!10}{637} & \cellcolor{gray!10}{49.49}\\
\hspace{1em}Treatment & 650 & 50.51\\
\addlinespace[0.3em]
\multicolumn{3}{l}{\textbf{Question Set}}\\
\cellcolor{gray!10}{\hspace{1em}A} & \cellcolor{gray!10}{644} & \cellcolor{gray!10}{50.04}\\
\hspace{1em}B & 643 & 49.96\\
\addlinespace[0.3em]
\multicolumn{3}{l}{\textbf{Age}}\\
\cellcolor{gray!10}{\hspace{1em}18-24} & \cellcolor{gray!10}{108} & \cellcolor{gray!10}{8.39}\\
\hspace{1em}25-34 & 242 & 18.80\\
\cellcolor{gray!10}{\hspace{1em}35-44} & \cellcolor{gray!10}{254} & \cellcolor{gray!10}{19.74}\\
\hspace{1em}45-54 & 226 & 17.56\\
\cellcolor{gray!10}{\hspace{1em}55-64} & \cellcolor{gray!10}{242} & \cellcolor{gray!10}{18.80}\\
\hspace{1em}65-74 & 148 & 11.50\\
\cellcolor{gray!10}{\hspace{1em}75-84} & \cellcolor{gray!10}{57} & \cellcolor{gray!10}{4.43}\\
\hspace{1em}85-92 & 10 & 0.78\\
\addlinespace[0.3em]
\multicolumn{3}{l}{\textbf{Gender}}\\
\cellcolor{gray!10}{\hspace{1em}Male} & \cellcolor{gray!10}{598} & \cellcolor{gray!10}{47.01}\\
\hspace{1em}Female & 658 & 51.73\\
\cellcolor{gray!10}{\hspace{1em}Other/Refused} & \cellcolor{gray!10}{16} & \cellcolor{gray!10}{1.26}\\
\addlinespace[0.3em]
\multicolumn{3}{l}{\textbf{Education}}\\
\hspace{1em}H.S. or less & 362 & 28.46\\
\cellcolor{gray!10}{\hspace{1em}Some college no degree} & \cellcolor{gray!10}{283} & \cellcolor{gray!10}{22.25}\\
\hspace{1em}Undergraduate level degree & 461 & 36.24\\
\cellcolor{gray!10}{\hspace{1em}Graduate level degree} & \cellcolor{gray!10}{166} & \cellcolor{gray!10}{13.05}\\
\addlinespace[0.3em]
\multicolumn{3}{l}{\textbf{Race}}\\
\hspace{1em}White or Caucasian & 975 & 76.65\\
\cellcolor{gray!10}{\hspace{1em}Black or African American} & \cellcolor{gray!10}{164} & \cellcolor{gray!10}{12.89}\\
\hspace{1em}American Indian & 22 & 1.73\\
\cellcolor{gray!10}{\hspace{1em}Asian} & \cellcolor{gray!10}{56} & \cellcolor{gray!10}{4.40}\\
\hspace{1em}Other & 55 & 4.32\\
\addlinespace[0.3em]
\multicolumn{3}{l}{\textbf{Party ID}}\\
\cellcolor{gray!10}{\hspace{1em}Republican} & \cellcolor{gray!10}{540} & \cellcolor{gray!10}{42.59}\\
\hspace{1em}Democrat & 566 & 44.64\\
\cellcolor{gray!10}{\hspace{1em}Independent} & \cellcolor{gray!10}{162} & \cellcolor{gray!10}{12.78}\\
\bottomrule
\end{tabular}}
\endgroup{}

}

\end{table}%

\textsubscript{Source:
\href{https://isaiahespi.github.io/rea-paper/index.qmd.html}{Article
Notebook}}

\pagebreak

\section*{References}\label{bibliography}
\addcontentsline{toc}{section}{References}

\phantomsection\label{refs}
\begin{CSLReferences}{1}{1}
\bibitem[\citeproctext]{ref-abbate2020a}
Abbate, Andrea. 2020. {``39 {Ways Election Offices} Are {Responding} to
{COVID-19}.''}
\url{https://www.techandciviclife.org/covid-19-responses/}.

\bibitem[\citeproctext]{ref-atkeson2007}
Atkeson, Lonna Rae, and Kyle L. Saunders. 2007. {``The {Effect} of
{Election Administration} on {Voter Confidence}: {A Local Matter}?''}
\emph{PS: Political Science \& Politics} 40(4): 655--60.
doi:\href{https://doi.org/10.1017/S1049096507071041}{10.1017/S1049096507071041}.

\bibitem[\citeproctext]{ref-bowler2024}
Bowler, Shaun, and Todd Donovan. 2024. {``Confidence in {US Elections
After} the {Big Lie}.''} \emph{Political Research Quarterly} 77(1):
283--96.
doi:\href{https://doi.org/10.1177/10659129231206179}{10.1177/10659129231206179}.

\bibitem[\citeproctext]{ref-brennancenterforjustice2024}
Brennan Center for Justice. 2024. \emph{Local {Election Officials
Survey} --- {May} 2024 \textbar{} {Brennan Center} for {Justice}}.
Brennan Center Research Department.
\url{https://www.brennancenter.org/our-work/research-reports/local-election-officials-survey-may-2024}
(November 5, 2024).

\bibitem[\citeproctext]{ref-carter2024}
Carter, Luke, Ashlan Gruwell, J Quin Monson, and Kelly D Patterson.
2024. {``From {Confidence} to {Convenience}: {Changes} in {Voting
Systems}, {Donald Trump}, and {Voter Confidence}.''} \emph{Public
Opinion Quarterly} 88: 516--35.
doi:\href{https://doi.org/10.1093/poq/nfae034}{10.1093/poq/nfae034}.

\bibitem[\citeproctext]{ref-cikara2014}
Cikara, Mina, and Jay J Van Bavel. 2014. {``The {Neuroscience} of
{Intergroup Relations}.''} \emph{Perspectives on Psychological Science}
9(3): 245--74.

\bibitem[\citeproctext]{ref-claassen2008}
Claassen, Ryan L., David B. Magleby, J. Quin Monson, and Kelly D.
Patterson. 2008. {``{`{At Your Service}'}: {Voter Evaluations} of {Poll
Worker Performance}.''} \emph{American Politics Research} 36(4):
612--34.
doi:\href{https://doi.org/10.1177/1532673X08319006}{10.1177/1532673X08319006}.

\bibitem[\citeproctext]{ref-coll2024a}
Coll, Joseph A, and Christopher J Clark. 2024. {``A {Racial Model} of
{Electoral Reform}: {The Relationship} Between {Restrictive Voting
Policies} and {Voter Confidence} for {Black} and {White Voters}.''}
\emph{Public Opinion Quarterly} 88: 561--84.
doi:\href{https://doi.org/10.1093/poq/nfae032}{10.1093/poq/nfae032}.

\bibitem[\citeproctext]{ref-conde2020}
Conde, Ximena. 2020. {``Philly Area Counties Say Efforts to Recruit Poll
Workers for {Election Day} Are Paying Off.''} \emph{WHYY NPR}.
\url{https://whyy.org/articles/philly-area-counties-say-efforts-to-recruit-poll-workers-for-election-day-are-paying-off/}.

\bibitem[\citeproctext]{ref-cook2005}
Cook, Timothy E., and Paul Gronke. 2005. {``The {Skeptical American}:
{Revisiting} the {Meanings} of {Trust} in {Government} and {Confidence}
in {Institutions}.''} \emph{The Journal of Politics} 67(3): 784--803.
doi:\href{https://doi.org/10.1111/j.1468-2508.2005.00339.x}{10.1111/j.1468-2508.2005.00339.x}.

\bibitem[\citeproctext]{ref-cooter2024}
Cooter, Amy. 2024. {``Veterans and {Extremism}: {From Militias} to
{January} 6th and {Real Patriots} \textbar{} {Middlebury Institute} of
{International Studies} at {Monterey}.''}
\url{https://www.middlebury.edu/institute/academics/centers-initiatives/ctec/ctec-publications/veterans-and-extremism-militias-january-6th}.

\bibitem[\citeproctext]{ref-cooter2013}
Cooter, Amy B. 2013. {``Americanness, {Masculinity}, and {Whiteness}:
{How Michigan Militia Men Navigate Evolving Social Norms}.''} Thesis.
\url{http://deepblue.lib.umich.edu/handle/2027.42/98077}.

\bibitem[\citeproctext]{ref-corrigan2002}
Corrigan, Patrick W., David Rowan, Amy Green, Robert Lundin, Philip
River, Kyle Uphoff-Wasowski, Kurt White, and Mary Anne Kubiak. 2002.
{``Challenging {Two Mental Illness Stigmas}: {Personal Responsibility}
and {Dangerousness}.''} \emph{Schizophrenia Bulletin} 28(2): 293--309.
doi:\href{https://doi.org/10.1093/oxfordjournals.schbul.a006939}{10.1093/oxfordjournals.schbul.a006939}.

\bibitem[\citeproctext]{ref-daniller2019}
Daniller, Andrew M, and Diana C Mutz. 2019. {``The {Dynamics} of
{Electoral Integrity}: {A Three-Election Panel Study}.''} \emph{Public
Opinion Quarterly} 83(1): 46--67.
doi:\href{https://doi.org/10.1093/poq/nfz002}{10.1093/poq/nfz002}.

\bibitem[\citeproctext]{ref-doubek2024}
Doubek, James. 2024. {``States and Cities Beef up Security to Prepare
for Potential Election-Related Violence.''} \emph{NPR: 2024 Election}.
\url{https://www.npr.org/2024/11/04/nx-s1-5178083/national-guard-police-election-security}
(November 5, 2024).

\bibitem[\citeproctext]{ref-dunn2018}
Dunn, Amina. 2018. {``Elections in {America}: {Concerns Over Security},
{Divisions Over Expanding Access} to {Voting}.''}
\url{https://www.pewresearch.org/politics/2018/10/29/elections-in-america-concerns-over-security-divisions-over-expanding-access-to-voting/}.

\bibitem[\citeproctext]{ref-edlin2024}
Edlin, Ruby, and Lawrence Norden. 2024. {``Poll of {Election Officials
Finds Concerns About Safety}, {Political Interference} \textbar{}
{Brennan Center} for {Justice}.''}
\url{https://www.brennancenter.org/our-work/analysis-opinion/poll-election-officials-finds-concerns-about-safety-political}.

\bibitem[\citeproctext]{ref-ferrer2024}
Ferrer, Joshua, Daniel M Thompson, and Rachel Orey. 2024. \emph{Election
{Official Turnover Rates} from 2000--2024}. Bipartisan Policy Center.
\url{https://bipartisanpolicy.org/download/?file=/wp-content/uploads/2024/04/WEB_BPC_Elections_Admin_Turnover_R01.pdf}.

\bibitem[\citeproctext]{ref-giles2021}
Giles, Ben. 2021. {``Arizona {Recount Of} 2020 {Election Ballots Found
No Proof Of Corruption}.''} \emph{NPR: 2024 Election}.
\url{https://www.npr.org/2021/09/25/1040672550/az-audit}.

\bibitem[\citeproctext]{ref-hall2009}
Hall, Thad E., J. Quin Monson, and Kelly D. Patterson. 2009. {``The
{Human Dimension} of {Elections}: {How Poll Workers Shape Public
Confidence} in {Elections}.''} \emph{Political Research Quarterly}
62(3): 507--22.
doi:\href{https://doi.org/10.1177/1065912908324870}{10.1177/1065912908324870}.

\bibitem[\citeproctext]{ref-hall2007}
Hall, Thad, J. Quin Monson, and Kelly D. Patterson. 2007. {``Poll
{Workers} and the {Vitality} of {Democracy}: {An Early Assessment}.''}
\emph{PS: Political Science \& Politics} 40(4): 647--54.
doi:\href{https://doi.org/10.1017/S104909650707103X}{10.1017/S104909650707103X}.

\bibitem[\citeproctext]{ref-hardy2019}
Hardy, Molly M., Calvin R. Coker, Michelle E. Funk, and Benjamin R.
Warner. 2019. {``Which Ingroup, When? {Effects} of Gender, Partisanship,
Veteran Status, and Evaluator Identities on Candidate Evaluations.''}
\emph{Communication Quarterly} 67(2): 199--220.
doi:\href{https://doi.org/10.1080/01463373.2019.1573201}{10.1080/01463373.2019.1573201}.

\bibitem[\citeproctext]{ref-herndon2020}
Herndon, Astead W. 2020. {``{LeBron James} and a {Multimillion-Dollar
Push} for {More Poll Workers}.''} \emph{The New York Times: U.S.}
\url{https://www.nytimes.com/2020/08/24/us/politics/lebron-james-poll-workers.html}
(November 13, 2024).

\bibitem[\citeproctext]{ref-herrnson2009}
Herrnson, Paul S., Richard G. Niemi, and Michael J. Hanmer. 2009.
\emph{Voting Technology : The Not-so-Simple Act of Casting a Ballot}.
Washington, D.C: Brookings Institution Press.

\bibitem[\citeproctext]{ref-hipes2016}
Hipes, Crosby, Jeffrey Lucas, Jo C. Phelan, and Richard C. White. 2016.
{``The Stigma of Mental Illness in the Labor Market.''} \emph{Social
Science Research} 56: 16--25.
doi:\href{https://doi.org/10.1016/j.ssresearch.2015.12.001}{10.1016/j.ssresearch.2015.12.001}.

\bibitem[\citeproctext]{ref-hooghe2018}
Hooghe, Marc. 2018. {``Trust and {Elections}.''} In \emph{The {Oxford
Handbook} of {Social} and {Political Trust}}, ed. Eric M. Uslaner.
Oxford University Press, 0.
doi:\href{https://doi.org/10.1093/oxfordhb/9780190274801.013.17}{10.1093/oxfordhb/9780190274801.013.17}.

\bibitem[\citeproctext]{ref-jensen2022a}
Jensen, Michael, Elizabeth Yates, and Sheehan Kane. 2022.
\emph{Radicalization in the {Ranks}: {An Assessment} of the {Scope} and
{Nature} of {Criminal Extremism} in the {United States Military}}.
{National Consortium for the Study of Terrorism and Responses to
Terrorism (START): College Park}.
\url{https://www.start.umd.edu/pubs/Radicalization\%20in\%20the\%20Ranks_April\%202022.pdf}.

\bibitem[\citeproctext]{ref-kleykamp2015}
Kleykamp, Meredith, and Crosby Hipes. 2015. {``Coverage of {Veterans} of
the {Wars} in {Iraq} and {Afghanistan} in the {U}.{S}. {Media}.''}
\emph{Sociological Forum} 30(2): 348--68.
doi:\href{https://doi.org/10.1111/socf.12166}{10.1111/socf.12166}.

\bibitem[\citeproctext]{ref-kleykamp2018}
Kleykamp, Meredith, Crosby Hipes, and Alair MacLean. 2018. {``Who
{Supports U}.{S}. {Veterans} and {Who Exaggerates Their Support}?''}
\emph{Armed Forces \& Society} 44(1): 92--115.
doi:\href{https://doi.org/10.1177/0095327X16682786}{10.1177/0095327X16682786}.

\bibitem[\citeproctext]{ref-kleykamp2023}
Kleykamp, Meredith, Daniel Schwam, and Gilad Wenig. 2023. \emph{What
{Americans Think About Veterans} and {Military Service}: {Findings} from
a {Nationally Representative Survey}}. RAND Corporation.
\url{https://www.rand.org/pubs/research_reports/RRA1363-7.html}.

\bibitem[\citeproctext]{ref-levendusky2024}
Levendusky, Matthew, Shawn Patterson Jr., Michele Margolis, Yotam Ophir,
Dror Walter, and Kathleen Hall Jamieson. 2024. {``The {Long Shadow} of
the {Big Lie}: {How Beliefs} about the {Legitimacy} of the 2020
{Election Spill Over} onto {Future Elections}.''} \emph{Public Opinion
Quarterly}: nfae047.
doi:\href{https://doi.org/10.1093/poq/nfae047}{10.1093/poq/nfae047}.

\bibitem[\citeproctext]{ref-lincoln2024}
Lincoln, Sophie. 2024. {``Washoe {County} Staff Prepare for {Election
Day}, Announce New Safety Feature at Polls.''}
\url{https://mynews4.com/news/local/washoe-county-staff-prepare-for-election-day-announce-new-safety-feature-at-polls}
(November 5, 2024).

\bibitem[\citeproctext]{ref-loewenson2023}
Loewenson, Irene. 2023. {``Mattis Says Vets at {Jan}. 6 {Capitol} Riot
{`Don't Define the Military'}.''} \emph{Marine Corps Times: name}.
\url{https://www.marinecorpstimes.com/news/your-marine-corps/2023/11/06/mattis-says-vets-at-jan-6-capitol-riot-dont-define-the-military/}.

\bibitem[\citeproctext]{ref-maclean2014}
MacLean, Alair, and Meredith Kleykamp. 2014. {``Coming {Home}:
{Attitudes} Toward {U}.{S}. {Veterans Returning} from {Iraq}.''}
\emph{Social Problems} 61(1): 131--54.
doi:\href{https://doi.org/10.1525/sp.2013.12074}{10.1525/sp.2013.12074}.

\bibitem[\citeproctext]{ref-magni2024}
Magni, Gabriele, and Andrew Reynolds. 2024. {``Candidate {Identity} and
{Campaign Priming}: {Analyzing Voter Support} for {Pete Buttigieg}'s
{Presidential Run} as an {Openly Gay Man}.''} \emph{Political Research
Quarterly} 77(1): 184--98.
doi:\href{https://doi.org/10.1177/10659129231194325}{10.1177/10659129231194325}.

\bibitem[\citeproctext]{ref-maidenberg1996}
Maidenberg, David H. 1996. \emph{Recruiting {Poll Workers}}. Office of
Election Administration, Federal Election Commission.
\url{https://purl.fdlp.gov/GPO/gpo18585}.

\bibitem[\citeproctext]{ref-maricopacountyelectionsdepartment2022}
Maricopa County Elections Department. 2022. \emph{Correcting the
{Record}: {Maricopa County}'s {In-Depth Analysis} of the {Senate
Inquiry}}. Maricopa County, Arizona: {Maricopa County Elections
Department and Office of the Recorder}.
\url{https://elections.maricopa.gov/asset/jcr:a9e03750-0a8f-4162-859f-1d46ac54b485/Correcting\%20The\%20Record\%20-\%20January\%202022\%20Report.pdf}.

\bibitem[\citeproctext]{ref-mena2020}
Mena, Kelly. 2020. {``States Scramble to Recruit Thousands of Poll
Workers Amid Pandemic Shortage \textbar{} {CNN Politics}.''}
\url{https://www.cnn.com/2020/08/13/politics/poll-worker-shortage-2020-election/index.html}.

\bibitem[\citeproctext]{ref-milton2021}
Milton, Daniel, and Andrew Mines. 2021. \emph{{``{This} Is {War}:''}
{Examining Military Experience Among} the {Capitol Hill Siege
Participants}}. George Washington University.
doi:\href{https://doi.org/10.4079/poe.04.2021.00}{10.4079/poe.04.2021.00}.

\bibitem[\citeproctext]{ref-nadeau1993}
Nadeau, Richard, and André Blais. 1993. {``Accepting the {Election
Outcome}: {The Effect} of {Participation} on {Losers}' {Consent}.''}
\emph{British Journal of Political Science} 23(4): 553--63.
doi:\href{https://doi.org/10.1017/S0007123400006736}{10.1017/S0007123400006736}.

\bibitem[\citeproctext]{ref-nadeem2024}
Nadeem, Reem. 2024. {``Harris, {Trump Voters Differ Over Election
Security}, {Vote Counts} and {Hacking Concerns}.''}
\url{https://www.pewresearch.org/politics/2024/10/24/harris-trump-voters-differ-over-election-security-vote-counts-and-hacking-concerns/}
(November 5, 2024).

\bibitem[\citeproctext]{ref-nevadasecretaryofstate2023}
Nevada Secretary of State. 2023. {``Governor {Joe Lombardo}, {Secretary}
of {State Francisco V}. {Aguilar} Sign {Election Worker Protection Bill}
into Law.''}
\url{https://www.nvsos.gov/sos/Home/Components/News/News/3368/309?backlist=\%2Fsos}
(November 5, 2024).

\bibitem[\citeproctext]{ref-nflfootballoperations2022}
NFL Football Operations. 2022. {``Vet the {Vote}.''}
\url{https://operations.nfl.com/inside-football-ops/social-justice/vet-the-vote/}.

\bibitem[\citeproctext]{ref-pape2024}
Pape, Robert A., Keven G. Ruby, Kyle D. Larson, and Kentaro Nakamura.
2024. {``Understanding the {Impact} of {Military Service} on {Support}
for {Insurrection} in the {United States}.''} \emph{Journal of Conflict
Resolution}: 00220027241267216.
doi:\href{https://doi.org/10.1177/00220027241267216}{10.1177/00220027241267216}.

\bibitem[\citeproctext]{ref-powerthepolls2020}
Power the Polls. 2020. {``Power {The Polls Launches First-of-its-Kind
Effort} to {Recruit New Wave} of {Poll Workers} for {Election Day}.''}
\url{https://www.powerthepolls.org/press-release-2020-06-30}.

\bibitem[\citeproctext]{ref-ross2020}
Ross, Doug. 2020. {``Porter {County} Election Officials Recruit Students
to Work Polls.''} \emph{nwitimes.com}.
\url{https://www.nwitimes.com/news/local/porter/porter-newsletter/porter-county-election-officials-recruit-students-to-work-polls/article_b2f1aaf8-1e5f-550e-bbbe-113dcb679f7e.html}.

\bibitem[\citeproctext]{ref-sances2015}
Sances, Michael W., and Charles Stewart. 2015. {``Partisanship and
Confidence in the Vote Count: {Evidence} from {U}.{S}. National
Elections Since 2000.''} \emph{Electoral Studies} 40: 176--88.
doi:\href{https://doi.org/10.1016/j.electstud.2015.08.004}{10.1016/j.electstud.2015.08.004}.

\bibitem[\citeproctext]{ref-steinhauer2020}
Steinhauer, Jennifer. 2020. {``Veterans {Fortify} the {Ranks} of
{Militias Aligned With Trump}'s {Views}.''} \emph{The New York Times:
U.S.}
\url{https://www.nytimes.com/2020/09/11/us/politics/veterans-trump-protests-militias.html}.

\bibitem[\citeproctext]{ref-stewart2022}
Stewart, Charles, III. 2022. {``Trust in {Elections}.''} \emph{Daedalus}
151(4): 234--53.
doi:\href{https://doi.org/10.1162/daed_a_01953}{10.1162/daed\_a\_01953}.

\bibitem[\citeproctext]{ref-studentveteransofamerica2020}
Student Veterans of America. 2020. {``Volunteer to {Power The Polls}.''}
\url{https://studentveterans.org/news/power-the-polls/}.

\bibitem[\citeproctext]{ref-vanbavel2021}
Van Bavel, Jay J., and Dominic J. Packer. 2021. \emph{The Power of Us :
Harnessing Our Shared Identities to Improve Performance, Increase
Cooperation, and Promote Social Harmony}. First edition. New York:
Little, Brown Spark.

\bibitem[\citeproctext]{ref-wang2022}
Wang, Hansi Lo. 2022. {``The Midterm Elections Need Workers. {Teens},
Veterans and Lawyers Are Stepping Up.''} \emph{NPR: 2024 Election}.
\url{https://www.npr.org/2022/09/07/1120065844/midterms-elections-poll-worker-shortage-recruitment}.

\bibitem[\citeproctext]{ref-wetheveterans2022}
We The Veterans. 2022. {``Launch of {Vet} the {Vote}.''}
\url{https://vetthe.vote/blogs/news/launch-of-vet-the-vote}.

\bibitem[\citeproctext]{ref-wire2024}
Wire, Sarah D., Phillip M. Bailey, Mary Jo Pitzl, Trevor Hughes, Erik
Pfantz, John Wisely, and Deborah Barfield Berry. 2024. {``Counting Votes
Is Now a Dangerous Job: How It Feels for Frontline, Swing-State
Workers.''}
\url{https://www.usatoday.com/story/news/politics/elections/2024/10/28/election-workers-2024-hostility/75586254007/}
(November 5, 2024).

\bibitem[\citeproctext]{ref-xiao2016}
Xiao, Y. Jenny, Géraldine Coppin, and Jay J. Van Bavel. 2016.
{``Perceiving the {World Through Group-Colored Glasses}: {A Perceptual
Model} of {Intergroup Relations}.''} \emph{Psychological Inquiry} 27(4):
255--74. \url{https://www.jstor.org/stable/26159704} (October 26, 2024).

\bibitem[\citeproctext]{ref-xiao2012}
Xiao, Y. Jenny, and Jay J. Van Bavel. 2012. {``See {Your Friends Close}
and {Your Enemies Closer}: {Social Identity} and {Identity Threat Shape}
the {Representation} of {Physical Distance}.''} \emph{Personality and
Social Psychology Bulletin} 38(7): 959--72.
doi:\href{https://doi.org/10.1177/0146167212442228}{10.1177/0146167212442228}.

\end{CSLReferences}




\end{document}
