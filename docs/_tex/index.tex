% Options for packages loaded elsewhere
\PassOptionsToPackage{unicode,pdfwindowui,pdfpagemode=FullScreen}{hyperref}
\PassOptionsToPackage{hyphens}{url}
\PassOptionsToPackage{dvipsnames,svgnames,x11names}{xcolor}
%
\documentclass[
  12pt,
  letterpaper,
]{article}

\usepackage{amsmath,amssymb}
\usepackage{iftex}
\ifPDFTeX
  \usepackage[T1]{fontenc}
  \usepackage[utf8]{inputenc}
  \usepackage{textcomp} % provide euro and other symbols
\else % if luatex or xetex
  \usepackage{unicode-math}
  \defaultfontfeatures{Scale=MatchLowercase}
  \defaultfontfeatures[\rmfamily]{Ligatures=TeX,Scale=1}
\fi
\usepackage{lmodern}
\ifPDFTeX\else  
    % xetex/luatex font selection
    \setmainfont[]{TeX Gyre Termes}
  \setmathfont[]{TeX Gyre Termes Math}
\fi
% Use upquote if available, for straight quotes in verbatim environments
\IfFileExists{upquote.sty}{\usepackage{upquote}}{}
\IfFileExists{microtype.sty}{% use microtype if available
  \usepackage[]{microtype}
  \UseMicrotypeSet[protrusion]{basicmath} % disable protrusion for tt fonts
}{}
\makeatletter
\@ifundefined{KOMAClassName}{% if non-KOMA class
  \IfFileExists{parskip.sty}{%
    \usepackage{parskip}
  }{% else
    \setlength{\parindent}{0pt}
    \setlength{\parskip}{6pt plus 2pt minus 1pt}}
}{% if KOMA class
  \KOMAoptions{parskip=half}}
\makeatother
\usepackage{xcolor}
\usepackage[margin=1in]{geometry}
\setlength{\emergencystretch}{3em} % prevent overfull lines
\setcounter{secnumdepth}{-\maxdimen} % remove section numbering


\providecommand{\tightlist}{%
  \setlength{\itemsep}{0pt}\setlength{\parskip}{0pt}}\usepackage{longtable,booktabs,array}
\usepackage{calc} % for calculating minipage widths
% Correct order of tables after \paragraph or \subparagraph
\usepackage{etoolbox}
\makeatletter
\patchcmd\longtable{\par}{\if@noskipsec\mbox{}\fi\par}{}{}
\makeatother
% Allow footnotes in longtable head/foot
\IfFileExists{footnotehyper.sty}{\usepackage{footnotehyper}}{\usepackage{footnote}}
\makesavenoteenv{longtable}
\usepackage{graphicx}
\makeatletter
\newsavebox\pandoc@box
\newcommand*\pandocbounded[1]{% scales image to fit in text height/width
  \sbox\pandoc@box{#1}%
  \Gscale@div\@tempa{\textheight}{\dimexpr\ht\pandoc@box+\dp\pandoc@box\relax}%
  \Gscale@div\@tempb{\linewidth}{\wd\pandoc@box}%
  \ifdim\@tempb\p@<\@tempa\p@\let\@tempa\@tempb\fi% select the smaller of both
  \ifdim\@tempa\p@<\p@\scalebox{\@tempa}{\usebox\pandoc@box}%
  \else\usebox{\pandoc@box}%
  \fi%
}
% Set default figure placement to htbp
\def\fps@figure{htbp}
\makeatother
% definitions for citeproc citations
\NewDocumentCommand\citeproctext{}{}
\NewDocumentCommand\citeproc{mm}{%
  \begingroup\def\citeproctext{#2}\cite{#1}\endgroup}
\makeatletter
 % allow citations to break across lines
 \let\@cite@ofmt\@firstofone
 % avoid brackets around text for \cite:
 \def\@biblabel#1{}
 \def\@cite#1#2{{#1\if@tempswa , #2\fi}}
\makeatother
\newlength{\cslhangindent}
\setlength{\cslhangindent}{1.5em}
\newlength{\csllabelwidth}
\setlength{\csllabelwidth}{3em}
\newenvironment{CSLReferences}[2] % #1 hanging-indent, #2 entry-spacing
 {\begin{list}{}{%
  \setlength{\itemindent}{0pt}
  \setlength{\leftmargin}{0pt}
  \setlength{\parsep}{0pt}
  % turn on hanging indent if param 1 is 1
  \ifodd #1
   \setlength{\leftmargin}{\cslhangindent}
   \setlength{\itemindent}{-1\cslhangindent}
  \fi
  % set entry spacing
  \setlength{\itemsep}{#2\baselineskip}}}
 {\end{list}}
\usepackage{calc}
\newcommand{\CSLBlock}[1]{\hfill\break\parbox[t]{\linewidth}{\strut\ignorespaces#1\strut}}
\newcommand{\CSLLeftMargin}[1]{\parbox[t]{\csllabelwidth}{\strut#1\strut}}
\newcommand{\CSLRightInline}[1]{\parbox[t]{\linewidth - \csllabelwidth}{\strut#1\strut}}
\newcommand{\CSLIndent}[1]{\hspace{\cslhangindent}#1}

% -----------------------
% CUSTOM PREAMBLE STUFF
% -----------------------

% -----------------
% Title block stuff
% -----------------

% Abstract
\usepackage[runin]{abstract}
\renewcommand{\abstractnamefont}{\sffamily\small\bfseries}
\renewcommand{\abstracttextfont}{\sffamily\small}
\setlength{\absleftindent}{5pt}
\setlength{\absrightindent}{\absleftindent}

% Title
\usepackage{titling}
\pretitle{\par\begin{flushleft}\LARGE\sffamily\bfseries}
\posttitle{\par\end{flushleft}\vskip 10pt}

% Keywords
\newenvironment{keywords}
{\small\sffamily{\sffamily\small\bfseries{Keywords.}}}

% Authors
\usepackage{orcidlink}  % Create automatic ORCID icons/links
%\renewcommand{\and}{\end{tabular} \hskip 3em \begin{tabular}[t]{@{\hspace{0em}}l@{}}}
\preauthor{\begin{flushleft}
           \lineskip 1.5em}
\postauthor{\end{flushleft}}

% ------------------
% Section headings
% ------------------
\usepackage{titlesec}
\titleformat*{\section}{\Large\sffamily\bfseries\raggedright}
\titleformat*{\subsection}{\large\sffamily\bfseries\raggedright}
\titleformat*{\subsubsection}{\normalsize\sffamily\bfseries\raggedright}
\titleformat*{\paragraph}{\small\sffamily\bfseries\raggedright}

%\titlespacing{<command>}{<left>}{<before-sep>}{<after-sep>}
% Starred version removes indentation in following paragraph
\titlespacing*{\section}{0em}{2em}{0.1em}
\titlespacing*{\subsection}{0em}{1.25em}{0.1em}
\titlespacing*{\subsubsection}{0em}{0.75em}{0em}

% ------------------
% Headers/Footers
% ------------------
\usepackage{fancyhdr}
\pagestyle{fancy}
\fancyhf{}
\fancyhead[L,C,R]{}
\fancyfoot[L,C]{}
\fancyfoot[R]{\thepage}
\renewcommand{\headrulewidth}{1pt}
\fancypagestyle{plain}{%
    \renewcommand{\headrulewidth}{0pt}%
    \fancyhf{}%
    \fancyfoot[R]{\thepage}%
}
\renewcommand\footnoterule{\rule{\linewidth}{0.1pt}\vspace{5pt}}

% ------------------
% Captions
% ------------------
\usepackage[labelfont=bf,labelsep=period]{caption}
\captionsetup[figure]{font=footnotesize,justification=raggedright,singlelinecheck=false,format=hang}


% ---------------------------
% END CUSTOM PREAMBLE STUFF
% ---------------------------
\usepackage{booktabs}
\usepackage{longtable}
\usepackage{array}
\usepackage{multirow}
\usepackage{wrapfig}
\usepackage{float}
\usepackage{colortbl}
\usepackage{pdflscape}
\usepackage{tabu}
\usepackage{threeparttable}
\usepackage{threeparttablex}
\usepackage[normalem]{ulem}
\usepackage{makecell}
\usepackage{xcolor}
\makeatletter
\@ifpackageloaded{caption}{}{\usepackage{caption}}
\AtBeginDocument{%
\ifdefined\contentsname
  \renewcommand*\contentsname{Table of contents}
\else
  \newcommand\contentsname{Table of contents}
\fi
\ifdefined\listfigurename
  \renewcommand*\listfigurename{List of Figures}
\else
  \newcommand\listfigurename{List of Figures}
\fi
\ifdefined\listtablename
  \renewcommand*\listtablename{List of Tables}
\else
  \newcommand\listtablename{List of Tables}
\fi
\ifdefined\figurename
  \renewcommand*\figurename{Figure}
\else
  \newcommand\figurename{Figure}
\fi
\ifdefined\tablename
  \renewcommand*\tablename{Table}
\else
  \newcommand\tablename{Table}
\fi
}
\@ifpackageloaded{float}{}{\usepackage{float}}
\floatstyle{ruled}
\@ifundefined{c@chapter}{\newfloat{codelisting}{h}{lop}}{\newfloat{codelisting}{h}{lop}[chapter]}
\floatname{codelisting}{Listing}
\newcommand*\listoflistings{\listof{codelisting}{List of Listings}}
\makeatother
\makeatletter
\makeatother
\makeatletter
\@ifpackageloaded{caption}{}{\usepackage{caption}}
\@ifpackageloaded{subcaption}{}{\usepackage{subcaption}}
\makeatother

\usepackage{bookmark}

\IfFileExists{xurl.sty}{\usepackage{xurl}}{} % add URL line breaks if available
\urlstyle{same} % disable monospaced font for URLs
\hypersetup{
  pdftitle={title},
  colorlinks=true,
  linkcolor={blue},
  filecolor={Maroon},
  citecolor={Blue},
  urlcolor={RoyalBlue},
  pdfcreator={LaTeX via pandoc}}



\title{title}
% subtitles do not seem to work with article class?
%%\usepackage{etoolbox}
%\makeatletter
%\providecommand{\subtitle}[1]{% add subtitle to \maketitle
%  \apptocmd{\@title}{\par {\large #1 \par}}{}{}
%}
%\makeatother
%%\subtitle{subtitle}

\author{
{\bfseries \normalsize Isaiah Espinoza}%
 \\%
 \small University of Maryland, Department of Government and
Politics \\%
{\footnotesize \url{gespinoz@umd.edu}} \\\vspace{10pt}
}

\predate{}
\postdate{}
\date{}
\begin{document}

% for some reason this does not work in header
\renewcommand{\abstractname}{Abstract.}

% add the short title to the fancy header
\fancyhead[R]{A Short title}
\fancyhead[L]{Isaiah}

\maketitle
%\noindent \rule{\linewidth}{.5pt}
%\noindent \rule{\linewidth}{.5pt}


\pagebreak

\subsection{Introduction}\label{introduction}

Election administration officials make efforts to sustain public trust
and confidence in the fairness and accuracy of elections, and attempt to
boost such confidence where it may be deprived. Concerns for safety have
developed among election staff and voters in more recent elections.
Regular measures are taken to enhance the
\emph{trustworthiness}\footnote{I adopt a distinction made between
  \emph{trustworthiness} and \emph{trust} in elections
  (\citeproc{ref-stewart2022}{Stewart 2022}). The ``\emph{worthiness}''
  of one's trust in the conduct and administration of elections is based
  on the extent that outcomes of an election reasonably follow the rules
  prescribed and can be adjudicated as such. To put it briefly,
  trustworthiness is built by the structure, procedures, and practices
  of the institution, in this case election administration. The public's
  trust in elections, however, is amendable to an indefinite number of
  factors that may be unrelated to the formal structure or procedures of
  election administration. To illustrate, I can recognize that my car is
  trustworthy prior to ever driving it because it is structurally sound;
  it passes whatever criteria upon inspection, it is sufficiently
  fueled, and appears to be in generally good working order. I have
  every reason to deem it \emph{worthy of my trust}. However, I don't
  trust my car because I am sure it is haunted.} of the electoral
process through practices meant to improve the conduct, transparency, or
overall administration of elections in the United States.

Although election officials undertake great efforts to enhance the
trustworthiness of election administration, public \emph{trust} in
elections is a psychological construct influenced by many things outside
of election official control such as partisanship or elite rhetoric
(\citeproc{ref-hooghe2018}{Hooghe 2018};
\citeproc{ref-sances2015}{Sances and Stewart 2015}). Moreover, a
person's evaluation of the election in hindsight is often influenced by
the election outcome itself (\citeproc{ref-daniller2019}{Daniller and
Mutz 2019}; \citeproc{ref-stewart2022}{Stewart 2022}). Thus, measures
taken by election officials can be undermined, trivialized, or made
irrelevant depending on how one feels after the election results have
come out.

Such volatile attitudes and evaluations post-election can leave a
lasting impression that election officials must contend with upon the
next election cycle (\citeproc{ref-bowler2024}{Bowler and Donovan 2024};
\citeproc{ref-levendusky2024}{Levendusky et al. 2024}). For instance, we
have witnessed many people's outright refusal to accept the 2020 U.S.
election results as legitimate despite consistent review of the evidence
confirming the results as fair and accurate. Such a case demonstrates
that public trust in elections is, at best, only partial to
trustworthiness of election administration in the United
States\footnote{There's little that could justify a conceptual
  distinction between public trust pre-election and public trust
  post-election. The temporal element renders the difference between
  pre- and post-election more operative than conceptual at a
  foundational level.}.

Even though election officials can do a lot to secure election
integrity, there's not much they can do to cement public confidence
after election night passes. At best, election officials can ease public
insecurities prior to election night.

One point of contention that election officials have faced in the past
regard evaluation of election workers. Previous literature has focused
on how voter interaction with election workers
(\citeproc{ref-claassen2008}{Claassen et al. 2008}), or the voter
experience generally (\citeproc{ref-atkeson2007}{Atkeson and Saunders
2007}), influences evaluations of election administration. As such,
election worker competency has been examined as a factor significant to
evaluations of performance of elections (\citeproc{ref-hall2007}{T.
Hall, Monson, and Patterson 2007}; \citeproc{ref-hall2009}{T. E. Hall,
Quin Monson, and Patterson 2009}). However, considering that individual
perceptions and preconceived notions play a huge role in cognition
(\citeproc{ref-cikara2014}{Cikara and Bavel 2014};
\citeproc{ref-vanbavel2021}{Van Bavel and Packer 2021}), it is
reasonable to expect that the group an election worker hails from would
be an important influence upon the voter's evaluation of the electoral
process.

Supposing such is the case, we can expect that information about
\emph{who} (i.e., which groups) election officials are targeting in
publicized recruitment efforts would lessen particular election
insecurity, and concurrently, boost confidence. That is to say, it is
reasonable to expect that telling people \emph{who} will be working and
volunteering as election staff would ease election insecurity, and
therefore improve confidence that the election will be conducted fairly,
accurately, and safe for all involved.

In this paper, I report results from a recent survey experiment
administered to test whether publicized efforts to recruit veterans to
work as election staff and volunteers would improve public trust in
elections and ease election insecurity. Results of the survey experiment
support the notion that emphasizing veterans as the target of election
worker recruitment efforts eases pre-election insecurity to a limited
extent. Expectations of electoral fraud and concerns for voter safety
were lower among those who read a fabricated announcement that veterans
are being recruited to work as election staff and volunteers compared to
those who read a control vignette where veterans were not mentioned.
Notable is that there was a significant difference in confidence among
those in the treatment condition who believe that results of the 2020
election were illegitimate.

This paper is structured as follows. First, I provide a brief background
on public trust in election administration. I synthesize a review of
relevant literature with a focus on how political and social science has
conceptualized and ascertained public trust in elections. Next, I supply
reasoning for why military veterans are singled out as the relevant
subset of the population in this study. I then explicitly provide the
simple theory and testable hypotheses of the study before moving on to
describe the design of the survey, the measurement instruments therein,
and the conceptual and operational definitions of the variables of
interest. I explain my reasoning and method for constructing the primary
dependent variable, which I broadly refer to as \emph{confidence in
elections}. I preview the different methods of analysis alongside
presentation of the results. Before concluding, I dedicate time to
discuss the limitations of the study and lessons learned. I close by
offering my interpretation of the results and suggest potential avenues
for future inquiry.

\subsection{Background: Election Administration and Public
Confidence}\label{background-election-administration-and-public-confidence}

Election officials have tried hard to inspire confidence in the
administration and conduct of elections by improving the degree to which
elections are trustworthy. Development and implementation of procedures
such as post-election auditing of ballots and logic-and-accuracy testing
of ballot tabulation equipment are prominent examples adding to the long
history of efforts to enhance the trustworthiness of election
administration in the United States.

Prior to the year 2000, one of the main issues facing election
administration was recruiting enough election workers to volunteer at
the polls (i.e., poll workers) (\citeproc{ref-maidenberg1996}{Maidenberg
1996}). Election worker recruitment is still much of an issue in the
current era as it was then, perhaps worse
(\citeproc{ref-ferrer2024}{Ferrer, Thompson, and Orey 2024}). In
addition to ensuring election admin offices were adequately staffed, the
controversy of the 2000 general election made the public more attentive
to issues concerning the conduct and administration of elections. In
particular, voting technology (\citeproc{ref-herrnson2009}{Herrnson,
Niemi, and Hanmer 2009}) and election worker competence was a of
interest in election studies (\citeproc{ref-claassen2008}{Claassen et
al. 2008}; \citeproc{ref-hall2007}{T. Hall, Monson, and Patterson 2007};
\citeproc{ref-hall2009}{T. E. Hall, Quin Monson, and Patterson 2009}).
Following the passage of the Help America Vote Act in 2002, election
officials efforts to boost public confidence in the conduct and
administration of elections revolved primarily around the accuracy of
vote counts, ballot tabulation equipment or voting machines, the
commitment of election staff, and more
(\citeproc{ref-atkeson2007}{Atkeson and Saunders 2007}).

In 2024, election officials made valiant efforts to boost public
confidence elections within an intensified political climate that
appeared quite hostile to election officials
(\citeproc{ref-brennancenterforjustice2024}{Brennan Center for Justice
2024}; \citeproc{ref-edlin2024}{Edlin and Norden 2024}). Although
polling around the time indicated that most people thought that U.S.
elections would be run at least somewhat well
(\citeproc{ref-nadeem2024}{Nadeem 2024}), many election officials
nationwide took efforts to assuage the worry of those most skeptical.

Election anxiety was high in the lead up to the 2024 elections in the
United States. Concerns for voter safety and the prospect of political
violence remained prescient and compelled many local officials to
prepare for the worst (\citeproc{ref-doubek2024}{Doubek 2024};
\citeproc{ref-edlin2024}{Edlin and Norden 2024}). Election officials in
Washoe County, Nevada, installed panic buttons for election staff that
would alert a monitoring center to summon law enforcement
(\citeproc{ref-lincoln2024}{Lincoln 2024}). Nevada also passed a law
making it a felony to harass, threaten, or intimidate election workers
(\citeproc{ref-nevadasecretaryofstate2023}{Nevada Secretary of State
2023}). Leading up to election day, news outlets reported that election
work had become a seemingly dangerous job (\citeproc{ref-wire2024}{Wire
et al. 2024}). A Brennan Center survey report stated that,
``\ldots large numbers of election officials report having experienced
threats, abuse, or harassment for doing their jobs''
(\citeproc{ref-edlin2024}{Edlin and Norden 2024}). Concerns over the
fairness of elections and accuracy of vote counts intensified,
heightening concerns over the prospect of political violence and, in
turn, increased worry for the safety of voters and election workers
alike.

Suffice to say, pre-election anxiety consists of more than confidence in
fairness and accuracy of vote counts in light of added safety concerns.
It is not hard to recognize that increased tension in the pre-election
period makes for a volatile political environment. Sustaining
trustworthiness in election administration is only more difficult in an
environment where turnover of election workers increases and the
struggle to recruit volunteers worsens in light of safety concerns.
Since election worker performance is significant to public evaluation of
elections, added safety concerns that drive out election staff and repel
volunteers can only detract from trustworthiness of the institution.

\subsection{Literature Reivew}\label{literature-reivew}

Trust and confidence in the conduct of elections concerns aspects of
elections that fall squarely within the institution of election
administration. At this level, for instance, public trust is ascertained
by capturing assessments about the perceived accuracy of vote counts
(e.g., whether votes are/were counted as intended). Or in other words,
by the extent the public is confident in the accurate administration of
elections.

Assessing the public's \emph{trust} in elections has not been
straightforward, however. Inquiry into public trust in elections has
been approached by scholars of political science in many different ways
(\citeproc{ref-cook2005}{Cook and Gronke 2005};
\citeproc{ref-hardin2004}{Hardin 2004}), often distinguishable by the
scope of the research question and more or less constrained by the
particular conception of public trust. Quite often, trust in election
administration is conflated with trust in government \emph{writ} large,
government legitimacy, government or system responsiveness, or even
satisfaction with democracy (\citeproc{ref-daniller2019}{Daniller and
Mutz 2019}). At this level, not only is the level of public trust in
elections sometimes vague, but there's little consideration over the
difference between such attitudes pre-election and post-election. In
contrast, a considerable amount of research tends to conceive of public
trust in accordance with the institution in question
(\citeproc{ref-atkeson2007}{Atkeson and Saunders 2007};
\citeproc{ref-hooghe2018}{Hooghe 2018}).

Elections administration is just one institution a part of the larger
set of institutions which form the electoral system. As such, the
performance of the institution along with the rest ``\ldots lends
credibility to the outcome of an election: whether it is considered by
citizens and the international community to be fair and legitimate.''
(\citeproc{ref-stewart2022}{Stewart 2022, 236}). Intuitively, enhancing
public trust in elections would best be accomplished by enhancing the
\emph{trustworthiness} of the institution, i.e., consistently doing the
things that election officials already regularly do come election time.
However, trust and confidence in elections has become ever more
precarious over the last few election cycles. Especially considering
public polling data since 2000 shows that confidence that votes were, or
would be, counted as intended was in a consistent decline despite
efforts towards bolstering election integrity and trustworthiness
(\citeproc{ref-sances2015}{Sances and Stewart 2015}). This is even more
pronounced considering the role that partisanship has had on such
confidence over accuracy of vote count (\citeproc{ref-sances2015}{Sances
and Stewart 2015}; \citeproc{ref-stewart2022}{Stewart 2022}).

There's also stark difference in public trust before the election has
occurred compared to after, a phenomenon referred to as the
``winner-loser gap''; the ``winners'' are those who supported the
winning candidate and the ``losers'' are those who supported the losing
candidate. Much research has been dedicated to analyzing the sentiment
of electoral winners vs losers, and vice versa
(\citeproc{ref-daniller2019}{Daniller and Mutz 2019};
\citeproc{ref-nadeau1993}{Nadeau and Blais 1993}). Opinions of electoral
trust gathered after the election has occurred are limited considering
the well-recognized impact that the electoral outcome itself has on
feelings of public trust in elections
(\citeproc{ref-daniller2019}{Daniller and Mutz 2019}).

As such, it is questionable whether we can characterize public trust in
the pre-election period as the same trust after the election results
have come out. The former is \emph{anticipatory}---i.e., the kind that
is more or less anxious given the uncertainties surrounding the
election. The latter is \emph{empirical}---a judgement discerned in
hindsight after the experience of the election event has occurred.
Evaluations of one's own trust and confidence in election administration
based on the experience of voting is influenced by that very experience.
As discussed by Stewart (\citeproc{ref-stewart2022}{2022}), not only was
confidence in vote count accuracy influenced by the voter experience,
but so too were evaluations of election officials
(\citeproc{ref-stewart2022}{2022, 242--43}). Not to mention the
influence that the election results would also have on such evaluations.
In other words, evaluations of trust and confidence post-election are
influenced by one's interpretation of their experience. The subtle
difference is simply the degree of uncertainty one feels in anticipation
of the next election event. This study focuses primarily on that
\emph{anticipatory} kind of confidence, which speaks more to those
insecurities\footnote{Note that election insecurity is used in place of
  electoral confidence simply to emphasize the anticipatory nature of
  the attitudes in question. Election insecurity is taken to denote a
  lack of confidence.} based on perceptions of the institution's
trustworthiness than upon the particular voter experience.

Regardless of the measures taken by election officials to boost public
confidence in the \emph{trustworthiness} of election administration
(e.g., conducting audits, testing election machines), public
\emph{trust} and confidence in elections more generally is apt to shift
dramatically post-election based on factors such as partisanship, elite
rhetoric, particular state policies, and more
(\citeproc{ref-carter2024}{Carter et al. 2024};
\citeproc{ref-coll2024a}{Coll and Clark 2024};
\citeproc{ref-nadeau1993}{Nadeau and Blais 1993}). Moreover, prior
research has found that evaluation of election workers themselves are an
important factor when it comes to levels of public confidence in the
electoral process (\citeproc{ref-claassen2008}{Claassen et al. 2008};
\citeproc{ref-hall2007}{T. Hall, Monson, and Patterson 2007};
\citeproc{ref-hall2009}{T. E. Hall, Quin Monson, and Patterson 2009}).
Such studies focused on the quality of the voter experience with
reference to the interaction between voter and election worker.

Beyond the general competence of election workers, however, the quality
of the voter experience may be influenced merely by \emph{a priori}
impressions about \emph{who} comprises election staff and volunteers.
Political and other social science researchers have recognized for some
time the power that group identity can have over attitudes and
perception (\citeproc{ref-vanbavel2021}{Van Bavel and Packer 2021};
\citeproc{ref-xiao2016}{Xiao, Coppin, and Van Bavel 2016};
\citeproc{ref-xiao2012}{Xiao and Van Bavel 2012}). As such, we can
expect that information identifying the particular groups being
recruited to serve as election staff will be enough to improve trust and
confidence in election administration, and lessen expectations of
electoral fraud. The next section elaborates on why military veterans
are of particular interest in this regard.

\subsection{Why Veterans}\label{why-veterans}

Election officials are likely to be agnostic as to who dedicates their
time to civil service such as election work. Staffing issues have been
an issue since at least the 1990s (\citeproc{ref-ferrer2024}{Ferrer,
Thompson, and Orey 2024}; \citeproc{ref-maidenberg1996}{Maidenberg
1996}). In 2020, such efforts were made far more difficult by the
COVID-19 pandemic (\citeproc{ref-abbate2020a}{Abbate 2020};
\citeproc{ref-mena2020}{Mena 2020}). Sure, it makes sense to recruit
veterans, but no more than any other group. After all, ensuring election
offices are adequately staffed is everyone's problem. So there's no
special reason to target veterans for recruitment above other groups.
Indeed, there's no reason to discriminate recruitment efforts at all if
the point is purely to fill staffing vacancies. However the interest in
veterans as a special group to consider arose in light of increased and
sudden efforts to recruit veterans into election work.

The reason why veterans arose as a special subset of the population to
consider for this inquiry is because there was a sudden push to target
veterans for recruitment efforts that arose shortly after the events on
Capitol Hill on January 6th, 2021. After the 2020 election, large
efforts were made to recruit military veterans and their families to
work or volunteer as election staff
(\citeproc{ref-nflfootballoperations2022}{NFL Football Operations 2022};
\citeproc{ref-wetheveterans2022}{We The Veterans 2022}). Prior to that
point, young people were sometimes given special mention as targets of
election worker recruitment efforts (\citeproc{ref-herndon2020}{Herndon
2020}; \citeproc{ref-powerthepolls2020}{Power the Polls 2020}).
Generally, however, recruitment efforts cast a wide net, indiscriminate
of who applies (\citeproc{ref-conde2020}{Conde 2020};
\citeproc{ref-ross2020}{Ross 2020}).

One can speculate that the motivation to associate military veterans
with civic engagement and democracy may be intended to counter negative
perceptions and impressions given by the proportion of veteran service
members arrested for taking part in the events on January 6th
(\citeproc{ref-jensen2022a}{Jensen, Yates, and Kane 2022};
\citeproc{ref-loewenson2023}{Loewenson 2023};
\citeproc{ref-milton2021}{Milton and Mines 2021}). Especially with
regard to research demonstrating that willingness to support violent
efforts to overturn election results (in support of Trump) is, on
average, more common among veterans than among matched samples of
non-veterans (\citeproc{ref-pape2024}{Pape et al. 2024}). This is in
addition to a strengthened association portrayed in media outlets
between military veterans and militias
(\citeproc{ref-steinhauer2020}{Steinhauer 2020}). Prior research has
substantiated such a connection between veterans and militia groups. A.
Cooter (\citeproc{ref-cooter2024}{2024}) notes from her 3-year
ethnographic fieldwork among Michigan militia members that,
``\ldots approximately 40\% of militia leaders and 30\% of members had
previous military experience. Most of these veterans actively sought out
such groups, as opposed to being recruited by them''
(\citeproc{ref-cooter2024}{2024}; see also \citeproc{ref-cooter2013}{A.
B. Cooter 2013}). Thus, countering such associations by promoting a
different image of veterans to the mass public and veterans alike seems
like a reasonable motivation. Yet such speculation is just that.

That being said, the general public perception, attitudes, or even
stereotypes about military veterans are significant to consider.
Although military recruitment shows a downward trend as of late, public
perception of veterans are overwhelmingly positive
(\citeproc{ref-kleykamp2023}{Kleykamp, Schwam, and Wenig 2023}).
Veterans are a particularly potent group where mention of veteran status
seems to have a positive, calming, or nullifying effect on attitudes.
For instance, recent research shows that, during his campaign in the
2020 Democratic Primaries, Pete Buttigieg's military background
mitigated discrimination against him when he was presented as a veteran
married to man (\citeproc{ref-magni2024}{Magni and Reynolds 2024}).
Similar research has also found that a candidate's veteran status
affords them better evaluations regarding competency in particular issue
areas (e.g., war competence) (\citeproc{ref-hardy2019}{Hardy et al.
2019}). Moreover, veteran status seems to mitigate or nullify usual
stigmas associated with mental illness. That is to say, there is
negative stigma associated with mental illness
(\citeproc{ref-corrigan2002}{Corrigan et al. 2002}) and such stigma
incurs labor market discrimination (\citeproc{ref-hipes2016}{Hipes et
al. 2016}), but evidence suggests that veteran status overrides such
stigma and discrimination (\citeproc{ref-maclean2014}{MacLean and
Kleykamp 2014}). Or, in another light, mental illness is seemingly more
\emph{understandable} (i.e., permissible) for veterans given the
presumptive reasons for their mental strife. And media framing as such
plays an important role on public perception
(\citeproc{ref-kleykamp2015}{Kleykamp and Hipes 2015}).

\subsection{Theory and Hypotheses}\label{theory-and-hypotheses}

Considering the generally positive perception afforded to veterans by
the public, it is reasonable to suspect that efforts that promote
recruitment of veterans and their family members to work as election
staff and volunteers would boost public confidence in election
administration. Indeed, that is the primary hypothesis of this study.
Although citizens may vote in various ways across the country (e.g., by
mail, ballot drop box, in-person), the simple announcement that election
officials are engaging in efforts to recruit veterans to work as
election staff may boost confidence in elections administration
regardless of how, or whether, an individual plans to vote. Formally,

\begin{quote}
H\textsubscript{1}: Announcements that election officials are recruiting
miilitary service veterans to work as election staff and volunteers will
be associated with greater confidence in elections administration
compared to announcements that do not mention military veterans.
\end{quote}

I considered that veterans would elicit high degrees of approval and
support for whatever cause or issue presented. For example, I'd expect
high approval and support for recruiting veterans to baseball teams just
as much as for recruiting them to work as election staff. Likewise,
asking participants whether they supported programs intent on recruiting
anyone to work as election staff would garner support regardless of the
target group. Considering that support and admiration for veterans is
generally high among the population
(\citeproc{ref-kleykamp2023}{Kleykamp, Schwam, and Wenig 2023}), then
discerning the impact on one's confidence would require a survey
experiment designed to determine whether publicized recruitment efforts
targeting veterans as a group would have any special effect on
confidence in elections administration.

Since the results and events following the 2020 election loomed large in
anticipation of the U.S. 2024 general elections, it is reasonable to
take into account beliefs about legitimacy of the 2020 election. This
group who denies the legitimacy of the 2020 election results were
possibly the most distrusting of the 2024 election. However, given that
veteran service members are held in high regard, generally speaking,
announcing that veterans are being actively recruited to work as
election staff and volunteers is expected to positively influence
confidence in elections among those who remained firm in the belief that
the 2020 election results were illegitimate. I take into account that
legitimacy beliefs about the 2020 election and expectations of electoral
fraud are related to partisanship and include partisanship as a control
in the analysis.

In addition, I expect that both expectations of electoral fraud and
concerns for voter safety would lessen (shrink) when presented with
information that veterans are being actively recruited to work and
volunteer in election offices.

Finally, prior research has shown that confidence that one's own vote
would be counted as intended has been much stronger than confidence that
votes nationwide would be treated likewise
(\citeproc{ref-sances2015}{Sances and Stewart 2015};
\citeproc{ref-stewart2022}{Stewart 2022}). In other words, the public is
more trusting of elections closer to home. Taking into account that
confidence is lower for elections conducted outside of one's local area,
then information that veterans are being recruited to help conduct
elections in locations beyond one's locale may close this gap between
confidence in elections within one's local area and confidence in
elections elsewhere. Although it may be difficult to discern whether
recruiting veteran service members would improve confidence in elections
near one's local area in light of such hometown favoritism, a positive
effect on confidence in elections held beyond one's local area---where
confidence is already expectedly lower---can help discern the impact of
the group alone. Therefore, any observed disparity of confidence in
elections (i.e., the ``confidence gap'') between elections within one's
local area and elections beyond will be smaller when people are
presented with information about election official recruitment efforts
of veteran service members.

\subsection{Experiment Design and Survey
Measures}\label{experiment-design-and-survey-measures}

To test the theory that publicized efforts to recruit veterans to work
as election staff and volunteers would improve confidence in elections
and ease insecurity, a recent experiment was embedded in a survey
developed and conducted by the Center for Democracy and Civic Engagement
(CDCE) at the University of Maryland. The survey was fielded from August
29th, 2024 to September 18th, 2024 on a non-probability sample of 1,287
U.S. citizens 18 years of age or older. Respondents were randomly split
into either the treatment (\(n =\) 650) control (\(n =\) 637)
conditions\footnote{Demographic breakdown of the sample are included in
  Appendix B.}.

Survey participants read either a treatment or control vignette, which
was a fabricated news article about efforts in Maricopa County, AZ to
recruit election staff and volunteers for the 2024 general election. The
treatment vignette referred to a program designed to recruit veterans
and their family members and describes an interviewee ``Jordan Braxton''
as an Army veteran. The control vignette simply omitted any mention of
veterans and their family members, and didn't describe ``Jordan
Braxton'' as an Army veteran. Beyond those small differences and the
headlines, the article vignettes are identical. Therefore, effects can
be attributed to the information about veterans in the treatment
vignette\footnote{Complete text of treatment and control vignettes are
  included in the Appendix}.

It should be noted that Maricopa County, AZ was chosen as the setting of
the story in the vignette due to the increased scrutiny levied toward
election administration there after the 2020 election
(\citeproc{ref-giles2021}{Giles 2021};
\citeproc{ref-maricopacountyelectionsdepartment2022}{Maricopa County
Elections Department 2022}). Because of this, treatment effects are
potentially limited or constrained to attitudes concerning the location
specified in the vignette. A block of survey items asked specifically
about Maricopa County, AZ, followed by an identical block of items that
asked the same questions about one's local area. Later, I compare and
discuss results between the items pertaining to different settings.

Over the past two decades it became commonplace for national polls to
gauge public confidence in election administration (i.e., voter
confidence) by asking some variety of the question, ``How confident are
you that your vote {[}will be/was{]} counted as you intended in the most
recent election?'' (\citeproc{ref-hall2009}{T. E. Hall, Quin Monson, and
Patterson 2009}; \citeproc{ref-sances2015}{Sances and Stewart 2015};
\citeproc{ref-stewart2022}{Stewart 2022}). In addition, since 2008, the
Survey of the Performance of American Elections (SPAE) has included a
good number of relevant questions to more thoroughly assess trust and
confidence in election administration. Such questions inquire into the
voter experience with the institution more directly. This study borrows,
modifies, or takes inspiration from certain question items found within
the 2022 SPAE and other survey items from the Pew Research Center's 2018
American Trends Panel wave 38 (\citeproc{ref-dunn2018}{Dunn 2018}).

Note that, except for the item assessing voter confidence (i.e., votes
counted as voters intend), most other question items from the 2022 SPAE
inquire into the voter experience after the election has occurred. The
items in this study, however, all inquire into confidence over one's
expectations in anticipation of upcoming elections. In addition, other
survey questionnaires ask similar questions that specifically contrast
between a respondent's own local community, their state, and nationwide;
this is in addition, or complementary to, items that distinguish between
local area officials, state election officials, state government, and
the U.S. federal government. The survey of this study, instead, focused
on attitudes regarding election workers such as election officials,
staff, and volunteers specific to the location mentioned in the
vignette---Maricopa County, AZ---as well as specific to one's local
area\footnote{Note that I refer to the items as either `AZ items' or
  `local items' in order to distinguish the location in which they
  pertain.}.

The survey included items that assessed both trust in elections
administration and expectations of electoral fraud, referred to simply
as \emph{trust} and \emph{distrust}, respectively. Responses from these
two sets of items were utilized to construct the primary dependent
variable, \emph{confidence in elections}.

Prior to the reading the treatment or control vignette, participants
responded to distinct survey items assessing one's favorability to local
election officials (``In general, how favorable or unfavorable is your
impression of local election officials?'') and legitimacy of the 2020
election results (``Regardless of whom you supported in the 2020
election, do you think Joe Biden's election as president was legitimate,
or was he not legitimately elected?''). The former presented five
response options from ``Strongly favorable'' to ``Strongly
unfavorable''; the latter item presented a binary of either
``Legitimate'' or ``Not Legitimate''.

The conventional multiple item assessment of partisanship was included
and used to construct a variable of partisanship consisting of three
categories: Democrat, Republican, and Independent. In this case,
Independents are ``true'' Independents in that they do not identify with
either political party and do not ``lean'' toward either Republican or
Democratic parties. Self-described ``Independent leaners'' are re-coded
as falling into the partisan category in which they lean e.g.,
Independents who expressly report that they lean Republican are denoted
as Republican.

Finally, a pair of survey items assessed concerns for potential violence
and confidence in voter safety. The pair was asked once in regard to
Maricopa County, AZ and again with respect to one's local
area\footnote{Survey item wording and responses are included in Appendix
  A.}.

\subsubsection{Trust and Distrust in
Elections}\label{trust-and-distrust-in-elections}

A set of five survey items measured public trust in elections by
inquiring into the degree of confidence that votes will be counted as
voters intend, confidence that electoral systems are secure from
technological threats, perceived commitment of election staff,
confidence that outcomes will be fair, and confidence that the voting
process will be fair. Each item in this series presented four response
options from ``Not at all {[}confident/committed{]}'', ``Not too
{[}confident/committed{]}'', ``Somewhat {[}confident/committed{]}'', and
``Very {[}confident/committed{]}''. In total, five survey items measure
different aspects of trust in elections administration. The items
measuring confidence in elections include:

Another series of five survey items captured an individual's level of
\emph{distrust} in elections administration based on the extent to which
they \emph{expect electoral fraud} to occur. These items were prefaced
with the question, ``How likely do you think any or all of the following
will happen during this year´s elections in {[}Maricopa County, AZ/ your
local area{]}?'' Each item presented four response options: ``Not likely
at all'', ``Not too likely'', ``Somewhat likely'', and ``Very likely''.

\begin{table}

\caption{\label{tbl-items}Survey Items for Trust and Distrust in
Elections}

\centering{

\begingroup\fontsize{8}{10}\selectfont

\begin{tabu} to \linewidth {>{\raggedright\arraybackslash}p{8cm}>{\raggedright\arraybackslash}p{8cm}}
\toprule
Trust & Distrust\\
\midrule
How confident are you that votes in Maricopa County, AZ will be counted as voters intend in the elections this November? & There will be voter fraud, that is, people who are not eligible to vote will vote, or vote more than once\\
\addlinespace
How confident are you that election officials, their staff, and volunteers in Maricopa County, AZ will do a good job conducting the elections this November? & Many votes will not actually be counted\\
\addlinespace
How commited do you think election staff and volunteers in Maricopa County, AZ will be to making sure the elections are fair and accurate? & Many people will show up to vote and be told they are not eligible\\
\addlinespace
How confident are you that the voting process will be fair in Maricopa County, AZ? & A foreign country will tamper with the votes cast in this area to change the results\\
\addlinespace
How confident are you that election systems in Maricopa County, AZ will be secure from hacking and other technological threats? & Election officials in Maricopa County, Arizona will try to discourage some people from voting\\
\bottomrule
\end{tabu}
\endgroup{}

}

\end{table}%

Similar to the items capturing trust in elections
(Table~\ref{tbl-items}), I computed a summated score for each respondent
reflecting the extent of their distrust in elections.

Upon construction of the two summated scales of \emph{trust} and
\emph{distrust}, internal consistency reliability coefficient \(\alpha\)
were estimated for both and found to be adequate (\(\alpha\)
\textgreater{} 0.8); the trust in elections scale \(\alpha\) = 0.93 and
the distrust scale \(\alpha\) = 0.86. In other words, the items in each
of the scales were intercorrelated well enough for each to compose a
single scale presuming a unidimensional construct for each.

I expected item responses from these two scales to be inversely
correlated among the sample, but not mutually exclusive. Indeed, this is
what I find. Polychoric correlations between the trust and distrust
items negatively correlate as expected{[}\^{}See Appendix C{]}, as does
the correlation between the two sum score scales. Given the ordinal
nature of the variable items, I conducted a Spearman's rank correlation
test (\citeproc{ref-spearman1907}{Spearman 1907}) and found a negative
correlation of \(\rho = -0.49\) (\(95\%\) CI {[}-0.53, -0.44{]};
Kendall's \(\tau = -0.39\)) between scores on the two scales for the AZ
items (For the local item scale score correlation: \(\rho = -0.52\),
\(95\%\) CI {[}-0.56, -0.48{]}; Kendall's \(\tau = -0.39\)). The
negative correlation between the trust and distrust items and scores
makes intuitive sense.

\subsection{Data Specification}\label{data-specification}

The primary outcome of interest concerns public confidence in elections
administration compensated by one's distrust in elections, i.e., their
expectations of electoral fraud. Because trust and distrust are assessed
by multiple Likert items, conventional methods call for combining the
items of each set into separate Likert scales (also known as summated
rating scales). Since the item response options are not
symmetric---i.e., not bipolar with contrasting positive and negative
response options to include a neutral middle option---then the
\emph{trust in elections} scale could only be understood as a measure of
trust in the positive use of the term (i.e., unipolar). Likewise, the
expectation of fraud items fit to a scale that corresponds to a unipolar
measure of confident \emph{distrust}, so to speak. However, many of the
items comprising the trust scale directly oppose items capturing
distrust. For instance, one item from the trust scale has the respondent
report their confidence that votes will be counted as voter's intend,
whereas a separate item from the distrust scale has the the respondent
report their expectation that many votes will not be counted. Positive
responses to these two items directly contradict each other, and would
thus cancel out.

The relationship is that trust and distrust represent opposing degrees
of confidence along the same spectrum. However, despite the inherent
contrast, one can hold both positive degrees of trust and distrust,
feeling similarly confident in both respects. Accordingly, equivalent
degrees of trust and distrust cancel out, rendering one relatively more
\emph{insecure} about their expectations of the future despite the fact
that they may feel confident in either direction. When trust and
distrust are considered as reflective of one's confidence about future
expectations, then both can be placed along the same spectrum.
Therefore, a lack of confidence denotes insecurity.

Something may cause a person to have greater expectations that election
fraud will occur, which in turn will always lower their trust in
elections administration. In contrast, a person may feel a bit more
confident in the integrity of elections, but this confidence is
constrained by the extent to which they expect election fraud to occur.
This means that a person's baseline level of confidence in elections
admin must take into account their present degree of distrust---i.e.,
their expectations of electoral fraud.

Those who are, or are close to being, equally confident in both
directions would be the most \emph{insecure} about their expectations.
Having no trust in elections and no expectation that electoral fraud
will occur is functionally equivalent to having equal degree of both.
Quite literally, this reflects an uncertain feeling of ``it could go one
way or the other'' with respect to the integrity of elections
administration. Operationally, this would reflect a middling point on a
scale of \emph{confidence}.

In light of this, I reverse coded positive expectations of electoral
fraud and added responses from the distrust scale to the trust in
elections scale. That is, item responses on the trust scale ranged from
0 (``Not confident at all'') to 3 (``Very confident), whereas responses
on the distrust scale ranged from 0 (''Not likely at all'') to -3
(``Very likely''). Combining responses in this way resulted in an
overall measure of \emph{confidence in elections} to serve as the
dependent variable\footnote{The methodological concerns between the use
  of (unit weighted) sum or mean scores compared to the use of estimated
  factor scores for Likert scales composed of multiple Likert items are
  legion. The use of either computed sum scores or estimated factor
  scores as a general practice is continuously debated
  (\citeproc{ref-mcneish2023}{McNeish 2023};
  \citeproc{ref-mcneish2020}{McNeish and Wolf 2020};
  \citeproc{ref-widaman2022}{Widaman and Revelle 2022},
  \citeproc{ref-widaman2024}{2024}), yet there is no general consensus
  on the most appropriate method generally. The use of either appears to
  largely depend on the researcher's particular research objective,
  theoretical suppositions, and necessity. Another common concern is the
  inappropriate treatment of ordinal variables as interval which levy
  strong assumptions, e.g., each item contributes equally to the
  variable being measured and that ordinal response categories are
  equidistant as though on an interval scale
  (\citeproc{ref-wang2013}{Wang et al. 2013}; for contrast, see
  \citeproc{ref-robitzsch2020}{Robitzsch 2020}).}. On this scale,
positive values reflect higher confidence in elections, negative values
reflect greater distrust, and values closer to zero correspond to an
individual's degree of insecurity.

The advantage is that this permits a better test of the treatment effect
as theorized. Of course, it is easy enough to examine the effect of the
treatment on both trust and distrust separately. However, the theory
presents a hypothesis that the treatment will impose influence upon both
variables \emph{concurrently and in a positive direction} with respect
to the relationship between trust and distrust. That is to say,
confidence increases insofar as trust in elections increases and
distrust decreases; since distrust detracts from trust, then an accurate
measure of confidence must take that constraint into account.

\subsection{Results}\label{results}

I first review responses to the survey items capturing trust and
distrust separately by treatment condition. When distinguished by
experiment condition, trust in elections admin is noticeably higher for
those who read the treatment vignette. The difference is more pronounced
for some items over others. That being said, there is a definite
distinction between items pertaining to Maricopa County, AZ compared to
items that pertain to a respondent's local area. In fact, the treatment
appears to have hardly made a difference when the items concerned one's
local area.

\phantomsection\label{cell-fig-trustlikert}
\begin{figure}[h]

\caption{\label{fig-trustlikert}Trust in Election Administration by
Experiment Condition}

\centering{

\pandocbounded{\includegraphics[keepaspectratio]{index_files/figure-pdf/fig-trustlikert-1.png}}

}

\end{figure}%

Reviewing item responses distinguished by treatment condition reveals
particulars of the treatment effect (Figure~\ref{fig-trustlikert}).
Among the items that asked about Maricopa County, AZ, it is a bit easier
to see upon which items the treatment vignette had the greatest effect.
It appears that the main effect of the treatment is upon responses to
the item that measures confidence in the security of election
technology.

Regarding confidence in the commitment of election staff and volunteers,
a higher proportion of respondents in the treatment group selected
``Very confident'' over ``Somewhat confident'' compared to the control
group, but only for items concerning Maricopa County, AZ. Although
confidence in election workers was generally high across the board, the
shift from ``Somewhat'' to ``Very'' confident for items pertaining to
election worker \emph{commitment} to fairness and accuracy seems to
clearly speak to the group being recruited, i.e., veterans and family
members. However, the same visible shift in proportion from ``Somewhat''
to ``Very'' confident isn't observed when items asked about one's local
area, which undermines the suggestion that the treatment effect can be
cleanly attributed to respondent's feelings about veteran service
members and their families. It may be the case that respondents harbor
particular attitudes or sentiment about Maricopa County, AZ itself, or
there may be nothing uniquely special about Maricopa County, AZ
supposing that elections anywhere beyond one's local area generally
garner slightly less confidence.

What this does suggest, however, is that trust in the fairness and
accuracy of the electoral process is notably higher upon the notion that
military veterans will be directly engaged as election staff and
volunteers in places where such trust falters. The question turns to
whether this also lowers distrust, or the extent to which one expects
electoral fraud to occur.

\phantomsection\label{cell-fig-distrustlikert}
\begin{figure}[h]

\caption{\label{fig-distrustlikert}Disrust in Election Administration by
Experiment Condition}

\centering{

\pandocbounded{\includegraphics[keepaspectratio]{index_files/figure-pdf/fig-distrustlikert-1.png}}

}

\end{figure}%

Similar to the trust scale items, the items on the distrust scale were
associated with lower distrust in the treatment condition compared to
the control. A quick glance at Figure~\ref{fig-distrustlikert} shows
what appears to be a significant reduction in distrust, especially
regarding the expectations that many votes wouldn't be counted, many
voters would be turned away, and that election officials would
discourage voters from casting a ballot. Interestingly, the expectation
that there would be foreign interference in elections in Maricopa
County, AZ were pretty resilient against the treatment. And again,
responses for items pertaining to elections in one's local area are
practically indistinguishable between treatment and control.

\subsubsection{Confidence in elections}\label{confidence-in-elections}

Once trust and distrust were composed into unified variable of
confidence in elections, I then examined distribution of the dependent
variable followed by analysis of the effect of the treatment by
comparison of mean differences between experiment conditions. Recall
that on this scale, zero reflects insecurity (e.g., similar or equal
values of trust and distrust), whereas values diverging from zero
reflect confidence in either direction, trust or distrust respectively.
Positive values reflect trust in elections administration (e.g.,
elections will be fair and accurate), whereas negative values reflect
distrust in elections (i.e., expectation that election fraud will
occur). To retain a meaningful zero, I rescaled the sum scores to range
from -3 to 3\footnote{This range is mostly arbitrary, as a range from -1
  to 1 works much the same. When relying on simple sum total score, the
  scale ranges from -15 to 15. So a single point increase from 0 to 1
  may reflect a combination of a single ``Very confident'' response on a
  trust item compensated by a single ``Somewhat likely'' response on a
  distrust item (i.e., 3-2 = 1), or some other equivalent combination.
  Although the use of mean scores would place scores back onto the
  response scale metric (e.g., from 0 to 3, reflective of ``not at all''
  to ``very''), this results in a unipolar scale (from 0 to 3). The
  bipolar scale resulting from the composition of positive trust and
  negative distrust engenders meaning to zero and negative values. As
  such, composite mean scores become inappropriate and interpretation of
  scores would no longer be feasible. In other words, negative values on
  the scale for confidence in elections hold substantive meaning (e.g.,
  distrust, absent trust), which makes transforming scores to fit a
  unipolar scale inappropriate.} and then superimposed two histograms of
confidence scores: one derived from AZ items and another for scores
derived from local area items. Doing this is illustrative of the
relative distribution of confidence that differed depending on the
location of elections referred to in the survey items (e.g., Maricopa
County, AZ).

As revealed by review of the item responses, the distribution of
confidence differed depending on whether survey items pertained to
elections in Maricopa County, AZ, or elections within one's local area.

\phantomsection\label{cell-fig-dist}
\begin{figure}[H]

\caption{\label{fig-dist}Overlapping Distribution of Confidence in
Elections for AZ and Local Area}

\centering{

\pandocbounded{\includegraphics[keepaspectratio]{index_files/figure-pdf/fig-dist-1.png}}

}

\end{figure}%

Overall, the sample expressed more trust in elections relative to
distrust seeing as how the distribution of confidence is largely
positive (Figure~\ref{fig-dist}). However, there's a clear difference
between confidence in Maricopa county, AZ elections and elections within
one's local area. A paired t-test confirms that this difference was
statistically significant, though somewhat small (mean difference
\(= 0.325\), \(95\%\) CI \([0.27, 0.38]\), \(t(1286) = 11.61\),
\(p < .001\)). Nonetheless, respondents appeared less confident
regarding elections in Maricopa County, AZ, but held more confidence in
elections administration in their local area. This complements previous
research that demonstrates a similar bias favorable to one's local area
(\citeproc{ref-stewart2022}{Stewart 2022}).

Accordingly, when confidence levels are distinguished by experiment
condition, the treatment vignette only had an effect upon survey items
pertaining to elections in Maricopa County, AZ. Conducting an
independent samples t-test (i.e., Welch Two Sample t-test) suggests that
the effect of the treatment compared to the control condition is
positive, statistically significant, yet substantially pretty small.

\begin{table}[H]

\caption{\label{tbl-ttest}Mean Difference Effect of Treatment Compared
to Control Condition}

\centering{

\centering\begingroup\fontsize{9}{11}\selectfont

\begin{tabular}[t]{llllllllll}
\toprule
Place & $\bar{x}_{diff}$ & $\bar{x}_{treat}$ & $\bar{x}_{control}$ & $treat_{n}$ & $control_{n}$ & t & p & df & CI\\
\midrule
Maricopa County, AZ & 0.20 & 0.93 & 0.72 & 650.00 & 637.00 & 2.96 & 0.003** & 1283.73 & {}[ 0.07, 0.33]\\
Local Area & 0.06 & 1.18 & 1.12 & 650.00 & 637.00 & 0.87 & 0.383 & 1282.95 & {}[-0.08, 0.20]\\
\bottomrule
\multicolumn{10}{l}{\rule{0pt}{1em}\textit{Note: }}\\
\multicolumn{10}{l}{\rule{0pt}{1em}Welch Two Sample t-tests of difference between Confidence in Elections by Treatment Condition.}\\
\end{tabular}
\endgroup{}

}

\end{table}%

The results of Table~\ref{tbl-ttest} show that, on a scale ranging from
-3 to 3, the effect of the treatment is associated with a \(0.20\)
average difference in confidence in elections in Maricopa County, AZ,
compared to the control group. When the dependent variable is centered
to have a mean of \(0\) and standard deviation of \(1\), standardized
parameters permit interpretation of the treatment effect in terms of
standard deviations; the standardized difference in confidence between
treatment and control is \(0.16\) (CI \([0.06, 0.27]\)). Nothing is more
illustrative than graphs and lines, however.

\begin{figure}[!t]

\caption{\label{fig-coef1}Confidence in Elections by Experiment
Condition}

\begin{minipage}{\linewidth}

\subcaption{\label{fig-coef1-1}Confidence by Treatment Condition}

\centering{

\pandocbounded{\includegraphics[keepaspectratio]{index_files/figure-pdf/fig-coef1-1.png}}

}

\end{minipage}%
\newline
\begin{minipage}{\linewidth}

\subcaption{\label{fig-coef1-2}Trust and Distrust by Treatment
Condition}

\centering{

\pandocbounded{\includegraphics[keepaspectratio]{index_files/figure-pdf/fig-coef1-2.png}}

}

\end{minipage}%

\end{figure}%

Figure~\ref{fig-coef1} displays standard difference estimates of
confidence in elections by treatment condition (control condition as
reference). Figure~\ref{fig-coef1-1} displays two models for confidence
in elections in Maricopa County, AZ, and one's local area by treatment
condition; Figure~\ref{fig-coef1-2} displays four models for trust and
distrust in elections in AZ or one's local area by treatment condition

the treatment effect on both trust and distrust is clear. Although the
effect is only there when questions pertained to Maricopa County, AZ.
The treatment effect on trust and distrust is conditional on whether the
survey questions inquired about elections in Maricopa County, AZ.
Announcing efforts to recruit veterans and their families to work as
election staff and volunteers increases confidence in elections
administration in places outside of one's local area, but doesn't boost
confidence in elections within one's local area. However, bear in mind
that confidence in elections for one's local area is already higher than
for those outside. Thus, there's already less insecurity concerning
elections in one's local area, but publicized efforts to recruit
veterans to work as election staff and volunteers is a small but
positive step towards closing that gap in confidence for elections that
occur elsewhere.

I now turn to results examining those who held onto the belief that the
2020 election results were illegitimate. I think check to see if
recruiting veterans simply makes people feel safer at the polls.

\pagebreak

\subsection{Conclusion}\label{conclusion}

\subsubsection{Limitations}\label{limitations}

As far as I'm aware, both the items, their wording, and their
composition into measurement scales are unique to this study. This
renders the composite scores for trust in elections sample dependent and
limits comparability with other cross-sectional survey data.

\begin{itemize}
\tightlist
\item
  limitations: interpretations of the treatment effect are limited.

  \begin{itemize}
  \tightlist
  \item
    The sample may not be generalizable to the population.
  \item
    It's hard to say whether merely mentioning veterans is enough, as
    compared to explicitly naming veterans as the target of particular
    recruitment efforts.
  \end{itemize}
\item
  Interpretation is also limited considering that no other particular
  group, or groups are compared directly against the veteran treatment
  vignette, i.e., additional vignettes for other comparable groups.
\item
  Moreover, results are limited by the survey questions, questionnaire
  design, and experimental stimulus (i.e., vignette) for two reasons. -
  First, the vignettes and many survey questions ask about Maricopa
  County, AZ specifically. Adding a specific county in the vignette adds
  in a factor that cannot be accounted for without additional treatments
  that eliminate the setting as a potential influence. Moreover, the
  specificity of the setting adds in even more unexplained error---some
  people may have attitudes about the county in question, others won't,
  while others may be muffed to consider a random county in the U.S.
  they've know nothing about. Adding the county undermines the
  confidence that treatment effects are solely attributable to veterans
  to an unknown degree. - Second, questions ask about one's own local
  community in addition to identical questions which asked about
  Maricopa County, AZ. Consequently, many questions within the
  questionnaire were duplicates with that one differences. This isn't
  uncommon in surveys, however in this case, it lengthened the survey to
  a degree that likely resulted in a higher drop off rate. More
  importantly, the quality of responses were likely diminished to some
  unknown extent. Although it is possible to compare questions asked
  about Maricopa County, AZ to questions about one's local area, the
  quality of that comparison is limited by the unknown extent of fatigue
  induced by answering the same questions twice. It wasn't just that
  some questions were asked multiple times, almost all of the questions
  were duplicated; after completing a long series on Maricopa County,
  participants then answered the same questions about the local
  community. The rate to which participants dropped from the survey is
  \_\_\_, which suggests fatigue as an important factor. Survey fatigue
  is a known issue that should be taken into consideration. Comparison
  of treatment effects between the MC series and the local area series
  is undermined by the unknown influence that fatigue would have on
  response choices. Any differences couldn't confidently be attributed
  to ``my area'' vs Maricopa county. We also can't determine whether
  treatment effects are sustained when comparing the Maricopa County
  series to the local area series asked subsequently. We would have to
  assume the setting as irrelevant, which we can't reasonably do.
\end{itemize}

\pagebreak

\subsection{Appendix A: Survey Experiment Vignettes and Survey
Items}\label{sec-appendix-a}

\begin{figure}

\begin{minipage}{0.45\linewidth}

\subsubsection{Treatment Vignette}\label{treatment-vignette-1}

\textbf{Local Military Veterans Recruited for Election Jobs in Maricopa
County}\\
\strut \\
\strut ~~PHOENIX (AP) --- Election officials in Maricopa County,
Arizona, announced a program designed to recruit military veterans and
their family members from the community to serve as election
administrators, including election polling place workers, temporary
workers, and full-time staff. As the U.S. general elections in November
near, election officials must fill several thousand temporary positions
and hundreds of other open positions to ensure sufficient staffing for
the 2024 elections and beyond.\\
\strut \\
\strut ~~Army veteran Jordan Braxton just joined the elections
workforce. Jordan believes their role is important to ensuring a secure,
accurate, and transparent election, ``Many places are short on staff
this election cycle. I served my country in the Army, and I want to do
my part as a veteran and a citizen to ensure that everyone trusts the
process and the outcome of the election.''

\end{minipage}%
%
\begin{minipage}{0.10\linewidth}
~\end{minipage}%
%
\begin{minipage}{0.45\linewidth}

\subsubsection{Control Vignette}\label{control-vignette-1}

\textbf{Local Residents Recruited for Election Jobs in Maricopa
County}\\
\strut \\
\strut ~~PHOENIX (AP) ---Election officials in Maricopa County, Arizona,
announced a program to recruit members of the community to serve as
election administrators, including election polling place workers,
temporary workers, and full-time staff. As the U.S. general elections in
November near, election officials must fill several thousand temporary
positions and hundreds of other open positions to ensure sufficient
staffing for the 2024 elections and beyond.\\
\strut \\
\strut ~~Jordan Braxton just joined the elections workforce. Jordan
believes their role is important to ensuring a secure, accurate, and
transparent election, ``Many places are short on staff this election
cycle. I want to do my part as a citizen to ensure that everyone trusts
the process and the outcome of the election.''

\end{minipage}%

\end{figure}%

\pagebreak

\subsection{Appendix B: Sample Demographics}\label{sec-appendix-b}

The median age was 46 (mean age was 47). There was 51.7\% (n = 658)
women, 47\% (n = 598) men, and approximately 1.3\% who identified as
either Non-binary/third gender (n = 7) or preferred not to say (n = 9).
A large proportion of the sample identified as White or Caucasian (n =
975, 76.65\%), while all other non-White respondents comprised 27.83\%
of the sample. Those who held a graduate level degree (e.g., Master's,
Doctorate, or Professional level) comprised 13\% of the sample; those
with either degree at the Associate or Bachelor's level comprised
36.24\%, while 22.25\% had some college but no degree; and 28.46\% had
either a high school level or equivalent education or less than high
school. The largest proportion of the sample identified as Democrat at
44.64\%, followed by Republicans at 42.59\%. The proportion of true
Independents was 12.78\%.

\pagebreak

\pagebreak

\section*{References}\label{bibliography}
\addcontentsline{toc}{section}{References}

\phantomsection\label{refs}
\begin{CSLReferences}{1}{1}
\bibitem[\citeproctext]{ref-abbate2020a}
Abbate, Andrea. 2020. {``39 {Ways Election Offices} Are {Responding} to
{COVID-19}.''}
\url{https://www.techandciviclife.org/covid-19-responses/}.

\bibitem[\citeproctext]{ref-atkeson2007}
Atkeson, Lonna Rae, and Kyle L. Saunders. 2007. {``The {Effect} of
{Election Administration} on {Voter Confidence}: {A Local Matter}?''}
\emph{PS: Political Science \& Politics} 40(4): 655--60.
doi:\href{https://doi.org/10.1017/S1049096507071041}{10.1017/S1049096507071041}.

\bibitem[\citeproctext]{ref-bowler2024}
Bowler, Shaun, and Todd Donovan. 2024. {``Confidence in {US Elections
After} the {Big Lie}.''} \emph{Political Research Quarterly} 77(1):
283--96.
doi:\href{https://doi.org/10.1177/10659129231206179}{10.1177/10659129231206179}.

\bibitem[\citeproctext]{ref-brennancenterforjustice2024}
Brennan Center for Justice. 2024. \emph{Local {Election Officials
Survey} --- {May} 2024 \textbar{} {Brennan Center} for {Justice}}.
Brennan Center Research Department.
\url{https://www.brennancenter.org/our-work/research-reports/local-election-officials-survey-may-2024}
(November 5, 2024).

\bibitem[\citeproctext]{ref-carter2024}
Carter, Luke, Ashlan Gruwell, J Quin Monson, and Kelly D Patterson.
2024. {``From {Confidence} to {Convenience}: {Changes} in {Voting
Systems}, {Donald Trump}, and {Voter Confidence}.''} \emph{Public
Opinion Quarterly} 88: 516--35.
doi:\href{https://doi.org/10.1093/poq/nfae034}{10.1093/poq/nfae034}.

\bibitem[\citeproctext]{ref-cikara2014}
Cikara, Mina, and Jay J Van Bavel. 2014. {``The {Neuroscience} of
{Intergroup Relations}.''} \emph{Perspectives on Psychological Science}
9(3): 245--74.

\bibitem[\citeproctext]{ref-claassen2008}
Claassen, Ryan L., David B. Magleby, J. Quin Monson, and Kelly D.
Patterson. 2008. {``{`{At Your Service}'}: {Voter Evaluations} of {Poll
Worker Performance}.''} \emph{American Politics Research} 36(4):
612--34.
doi:\href{https://doi.org/10.1177/1532673X08319006}{10.1177/1532673X08319006}.

\bibitem[\citeproctext]{ref-coll2024a}
Coll, Joseph A, and Christopher J Clark. 2024. {``A {Racial Model} of
{Electoral Reform}: {The Relationship} Between {Restrictive Voting
Policies} and {Voter Confidence} for {Black} and {White Voters}.''}
\emph{Public Opinion Quarterly} 88: 561--84.
doi:\href{https://doi.org/10.1093/poq/nfae032}{10.1093/poq/nfae032}.

\bibitem[\citeproctext]{ref-conde2020}
Conde, Ximena. 2020. {``Philly Area Counties Say Efforts to Recruit Poll
Workers for {Election Day} Are Paying Off.''} \emph{WHYY NPR}.
\url{https://whyy.org/articles/philly-area-counties-say-efforts-to-recruit-poll-workers-for-election-day-are-paying-off/}.

\bibitem[\citeproctext]{ref-cook2005}
Cook, Timothy E., and Paul Gronke. 2005. {``The {Skeptical American}:
{Revisiting} the {Meanings} of {Trust} in {Government} and {Confidence}
in {Institutions}.''} \emph{The Journal of Politics} 67(3): 784--803.
doi:\href{https://doi.org/10.1111/j.1468-2508.2005.00339.x}{10.1111/j.1468-2508.2005.00339.x}.

\bibitem[\citeproctext]{ref-cooter2024}
Cooter, Amy. 2024. {``Veterans and {Extremism}: {From Militias} to
{January} 6th and {Real Patriots} \textbar{} {Middlebury Institute} of
{International Studies} at {Monterey}.''}
\url{https://www.middlebury.edu/institute/academics/centers-initiatives/ctec/ctec-publications/veterans-and-extremism-militias-january-6th}.

\bibitem[\citeproctext]{ref-cooter2013}
Cooter, Amy B. 2013. {``Americanness, {Masculinity}, and {Whiteness}:
{How Michigan Militia Men Navigate Evolving Social Norms}.''} Thesis.
\url{http://deepblue.lib.umich.edu/handle/2027.42/98077}.

\bibitem[\citeproctext]{ref-corrigan2002}
Corrigan, Patrick W., David Rowan, Amy Green, Robert Lundin, Philip
River, Kyle Uphoff-Wasowski, Kurt White, and Mary Anne Kubiak. 2002.
{``Challenging {Two Mental Illness Stigmas}: {Personal Responsibility}
and {Dangerousness}.''} \emph{Schizophrenia Bulletin} 28(2): 293--309.
doi:\href{https://doi.org/10.1093/oxfordjournals.schbul.a006939}{10.1093/oxfordjournals.schbul.a006939}.

\bibitem[\citeproctext]{ref-daniller2019}
Daniller, Andrew M, and Diana C Mutz. 2019. {``The {Dynamics} of
{Electoral Integrity}: {A Three-Election Panel Study}.''} \emph{Public
Opinion Quarterly} 83(1): 46--67.
doi:\href{https://doi.org/10.1093/poq/nfz002}{10.1093/poq/nfz002}.

\bibitem[\citeproctext]{ref-doubek2024}
Doubek, James. 2024. {``States and Cities Beef up Security to Prepare
for Potential Election-Related Violence.''} \emph{NPR: 2024 Election}.
\url{https://www.npr.org/2024/11/04/nx-s1-5178083/national-guard-police-election-security}
(November 5, 2024).

\bibitem[\citeproctext]{ref-dunn2018}
Dunn, Amina. 2018. {``Elections in {America}: {Concerns Over Security},
{Divisions Over Expanding Access} to {Voting}.''}
\url{https://www.pewresearch.org/politics/2018/10/29/elections-in-america-concerns-over-security-divisions-over-expanding-access-to-voting/}.

\bibitem[\citeproctext]{ref-edlin2024}
Edlin, Ruby, and Lawrence Norden. 2024. {``Poll of {Election Officials
Finds Concerns About Safety}, {Political Interference} \textbar{}
{Brennan Center} for {Justice}.''}
\url{https://www.brennancenter.org/our-work/analysis-opinion/poll-election-officials-finds-concerns-about-safety-political}.

\bibitem[\citeproctext]{ref-ferrer2024}
Ferrer, Joshua, Daniel M Thompson, and Rachel Orey. 2024. \emph{Election
{Official Turnover Rates} from 2000--2024}. Bipartisan Policy Center.
\url{https://bipartisanpolicy.org/download/?file=/wp-content/uploads/2024/04/WEB_BPC_Elections_Admin_Turnover_R01.pdf}.

\bibitem[\citeproctext]{ref-giles2021}
Giles, Ben. 2021. {``Arizona {Recount Of} 2020 {Election Ballots Found
No Proof Of Corruption}.''} \emph{NPR: 2024 Election}.
\url{https://www.npr.org/2021/09/25/1040672550/az-audit}.

\bibitem[\citeproctext]{ref-hall2009}
Hall, Thad E., J. Quin Monson, and Kelly D. Patterson. 2009. {``The
{Human Dimension} of {Elections}: {How Poll Workers Shape Public
Confidence} in {Elections}.''} \emph{Political Research Quarterly}
62(3): 507--22.
doi:\href{https://doi.org/10.1177/1065912908324870}{10.1177/1065912908324870}.

\bibitem[\citeproctext]{ref-hall2007}
Hall, Thad, J. Quin Monson, and Kelly D. Patterson. 2007. {``Poll
{Workers} and the {Vitality} of {Democracy}: {An Early Assessment}.''}
\emph{PS: Political Science \& Politics} 40(4): 647--54.
doi:\href{https://doi.org/10.1017/S104909650707103X}{10.1017/S104909650707103X}.

\bibitem[\citeproctext]{ref-hardin2004}
Hardin, Russell. 2004. \emph{Distrust}. New York: Russell Sage
Foundation.

\bibitem[\citeproctext]{ref-hardy2019}
Hardy, Molly M., Calvin R. Coker, Michelle E. Funk, and Benjamin R.
Warner. 2019. {``Which Ingroup, When? {Effects} of Gender, Partisanship,
Veteran Status, and Evaluator Identities on Candidate Evaluations.''}
\emph{Communication Quarterly} 67(2): 199--220.
doi:\href{https://doi.org/10.1080/01463373.2019.1573201}{10.1080/01463373.2019.1573201}.

\bibitem[\citeproctext]{ref-herndon2020}
Herndon, Astead W. 2020. {``{LeBron James} and a {Multimillion-Dollar
Push} for {More Poll Workers}.''} \emph{The New York Times: U.S.}
\url{https://www.nytimes.com/2020/08/24/us/politics/lebron-james-poll-workers.html}
(November 13, 2024).

\bibitem[\citeproctext]{ref-herrnson2009}
Herrnson, Paul S., Richard G. Niemi, and Michael J. Hanmer. 2009.
\emph{Voting Technology : The Not-so-Simple Act of Casting a Ballot}.
Washington, D.C: Brookings Institution Press.

\bibitem[\citeproctext]{ref-hipes2016}
Hipes, Crosby, Jeffrey Lucas, Jo C. Phelan, and Richard C. White. 2016.
{``The Stigma of Mental Illness in the Labor Market.''} \emph{Social
Science Research} 56: 16--25.
doi:\href{https://doi.org/10.1016/j.ssresearch.2015.12.001}{10.1016/j.ssresearch.2015.12.001}.

\bibitem[\citeproctext]{ref-hooghe2018}
Hooghe, Marc. 2018. {``Trust and {Elections}.''} In \emph{The {Oxford
Handbook} of {Social} and {Political Trust}}, ed. Eric M. Uslaner.
Oxford University Press, 0.
doi:\href{https://doi.org/10.1093/oxfordhb/9780190274801.013.17}{10.1093/oxfordhb/9780190274801.013.17}.

\bibitem[\citeproctext]{ref-jensen2022a}
Jensen, Michael, Elizabeth Yates, and Sheehan Kane. 2022.
\emph{Radicalization in the {Ranks}: {An Assessment} of the {Scope} and
{Nature} of {Criminal Extremism} in the {United States Military}}.
{National Consortium for the Study of Terrorism and Responses to
Terrorism (START): College Park}.
\url{https://www.start.umd.edu/pubs/Radicalization\%20in\%20the\%20Ranks_April\%202022.pdf}.

\bibitem[\citeproctext]{ref-kleykamp2015}
Kleykamp, Meredith, and Crosby Hipes. 2015. {``Coverage of {Veterans} of
the {Wars} in {Iraq} and {Afghanistan} in the {U}.{S}. {Media}.''}
\emph{Sociological Forum} 30(2): 348--68.
doi:\href{https://doi.org/10.1111/socf.12166}{10.1111/socf.12166}.

\bibitem[\citeproctext]{ref-kleykamp2023}
Kleykamp, Meredith, Daniel Schwam, and Gilad Wenig. 2023. \emph{What
{Americans Think About Veterans} and {Military Service}: {Findings} from
a {Nationally Representative Survey}}. RAND Corporation.
\url{https://www.rand.org/pubs/research_reports/RRA1363-7.html}.

\bibitem[\citeproctext]{ref-levendusky2024}
Levendusky, Matthew, Shawn Patterson Jr., Michele Margolis, Yotam Ophir,
Dror Walter, and Kathleen Hall Jamieson. 2024. {``The {Long Shadow} of
the {Big Lie}: {How Beliefs} about the {Legitimacy} of the 2020
{Election Spill Over} onto {Future Elections}.''} \emph{Public Opinion
Quarterly}: nfae047.
doi:\href{https://doi.org/10.1093/poq/nfae047}{10.1093/poq/nfae047}.

\bibitem[\citeproctext]{ref-lincoln2024}
Lincoln, Sophie. 2024. {``Washoe {County} Staff Prepare for {Election
Day}, Announce New Safety Feature at Polls.''}
\url{https://mynews4.com/news/local/washoe-county-staff-prepare-for-election-day-announce-new-safety-feature-at-polls}
(November 5, 2024).

\bibitem[\citeproctext]{ref-loewenson2023}
Loewenson, Irene. 2023. {``Mattis Says Vets at {Jan}. 6 {Capitol} Riot
{`Don't Define the Military'}.''} \emph{Marine Corps Times: name}.
\url{https://www.marinecorpstimes.com/news/your-marine-corps/2023/11/06/mattis-says-vets-at-jan-6-capitol-riot-dont-define-the-military/}.

\bibitem[\citeproctext]{ref-maclean2014}
MacLean, Alair, and Meredith Kleykamp. 2014. {``Coming {Home}:
{Attitudes} Toward {U}.{S}. {Veterans Returning} from {Iraq}.''}
\emph{Social Problems} 61(1): 131--54.
doi:\href{https://doi.org/10.1525/sp.2013.12074}{10.1525/sp.2013.12074}.

\bibitem[\citeproctext]{ref-magni2024}
Magni, Gabriele, and Andrew Reynolds. 2024. {``Candidate {Identity} and
{Campaign Priming}: {Analyzing Voter Support} for {Pete Buttigieg}'s
{Presidential Run} as an {Openly Gay Man}.''} \emph{Political Research
Quarterly} 77(1): 184--98.
doi:\href{https://doi.org/10.1177/10659129231194325}{10.1177/10659129231194325}.

\bibitem[\citeproctext]{ref-maidenberg1996}
Maidenberg, David H. 1996. \emph{Recruiting {Poll Workers}}. Office of
Election Administration, Federal Election Commission.
\url{https://purl.fdlp.gov/GPO/gpo18585}.

\bibitem[\citeproctext]{ref-maricopacountyelectionsdepartment2022}
Maricopa County Elections Department. 2022. \emph{Correcting the
{Record}: {Maricopa County}'s {In-Depth Analysis} of the {Senate
Inquiry}}. Maricopa County, Arizona: {Maricopa County Elections
Department and Office of the Recorder}.
\url{https://elections.maricopa.gov/asset/jcr:a9e03750-0a8f-4162-859f-1d46ac54b485/Correcting\%20The\%20Record\%20-\%20January\%202022\%20Report.pdf}.

\bibitem[\citeproctext]{ref-mcneish2023}
McNeish, Daniel. 2023. {``Psychometric Properties of Sum Scores and
Factor Scores Differ Even When Their Correlation Is 0.98: {A} Response
to {Widaman} and {Revelle}.''} \emph{Behavior Research Methods} 55(8):
4269--90.
doi:\href{https://doi.org/10.3758/s13428-022-02016-x}{10.3758/s13428-022-02016-x}.

\bibitem[\citeproctext]{ref-mcneish2020}
McNeish, Daniel, and Melissa Gordon Wolf. 2020. {``Thinking Twice about
Sum Scores.''} \emph{Behavior Research Methods} 52(6): 2287--2305.
doi:\href{https://doi.org/10.3758/s13428-020-01398-0}{10.3758/s13428-020-01398-0}.

\bibitem[\citeproctext]{ref-mena2020}
Mena, Kelly. 2020. {``States Scramble to Recruit Thousands of Poll
Workers Amid Pandemic Shortage \textbar{} {CNN Politics}.''}
\url{https://www.cnn.com/2020/08/13/politics/poll-worker-shortage-2020-election/index.html}.

\bibitem[\citeproctext]{ref-milton2021}
Milton, Daniel, and Andrew Mines. 2021. \emph{{``{This} Is {War}:''}
{Examining Military Experience Among} the {Capitol Hill Siege
Participants}}. George Washington University.
doi:\href{https://doi.org/10.4079/poe.04.2021.00}{10.4079/poe.04.2021.00}.

\bibitem[\citeproctext]{ref-nadeau1993}
Nadeau, Richard, and André Blais. 1993. {``Accepting the {Election
Outcome}: {The Effect} of {Participation} on {Losers}' {Consent}.''}
\emph{British Journal of Political Science} 23(4): 553--63.
doi:\href{https://doi.org/10.1017/S0007123400006736}{10.1017/S0007123400006736}.

\bibitem[\citeproctext]{ref-nadeem2024}
Nadeem, Reem. 2024. {``Harris, {Trump Voters Differ Over Election
Security}, {Vote Counts} and {Hacking Concerns}.''}
\url{https://www.pewresearch.org/politics/2024/10/24/harris-trump-voters-differ-over-election-security-vote-counts-and-hacking-concerns/}
(November 5, 2024).

\bibitem[\citeproctext]{ref-nevadasecretaryofstate2023}
Nevada Secretary of State. 2023. {``Governor {Joe Lombardo}, {Secretary}
of {State Francisco V}. {Aguilar} Sign {Election Worker Protection Bill}
into Law.''}
\url{https://www.nvsos.gov/sos/Home/Components/News/News/3368/309?backlist=\%2Fsos}
(November 5, 2024).

\bibitem[\citeproctext]{ref-nflfootballoperations2022}
NFL Football Operations. 2022. {``Vet the {Vote}.''}
\url{https://operations.nfl.com/inside-football-ops/social-justice/vet-the-vote/}.

\bibitem[\citeproctext]{ref-pape2024}
Pape, Robert A., Keven G. Ruby, Kyle D. Larson, and Kentaro Nakamura.
2024. {``Understanding the {Impact} of {Military Service} on {Support}
for {Insurrection} in the {United States}.''} \emph{Journal of Conflict
Resolution}: 00220027241267216.
doi:\href{https://doi.org/10.1177/00220027241267216}{10.1177/00220027241267216}.

\bibitem[\citeproctext]{ref-powerthepolls2020}
Power the Polls. 2020. {``Power {The Polls Launches First-of-its-Kind
Effort} to {Recruit New Wave} of {Poll Workers} for {Election Day}.''}
\url{https://www.powerthepolls.org/press-release-2020-06-30}.

\bibitem[\citeproctext]{ref-robitzsch2020}
Robitzsch, Alexander. 2020. {``Why {Ordinal Variables Can} ({Almost})
{Always Be Treated} as {Continuous Variables}: {Clarifying Assumptions}
of {Robust Continuous} and {Ordinal Factor Analysis Estimation
Methods}.''} \emph{Frontiers in Education} 5.
doi:\href{https://doi.org/10.3389/feduc.2020.589965}{10.3389/feduc.2020.589965}.

\bibitem[\citeproctext]{ref-ross2020}
Ross, Doug. 2020. {``Porter {County} Election Officials Recruit Students
to Work Polls.''} \emph{nwitimes.com}.
\url{https://www.nwitimes.com/news/local/porter/porter-newsletter/porter-county-election-officials-recruit-students-to-work-polls/article_b2f1aaf8-1e5f-550e-bbbe-113dcb679f7e.html}.

\bibitem[\citeproctext]{ref-sances2015}
Sances, Michael W., and Charles Stewart. 2015. {``Partisanship and
Confidence in the Vote Count: {Evidence} from {U}.{S}. National
Elections Since 2000.''} \emph{Electoral Studies} 40: 176--88.
doi:\href{https://doi.org/10.1016/j.electstud.2015.08.004}{10.1016/j.electstud.2015.08.004}.

\bibitem[\citeproctext]{ref-spearman1907}
Spearman, Charles. 1907. {``Demonstration of {Formulæ} for {True
Measurement} of {Correlation}.''} \emph{The American Journal of
Psychology} 18(2): 161--69.
doi:\href{https://doi.org/10.2307/1412408}{10.2307/1412408}.

\bibitem[\citeproctext]{ref-steinhauer2020}
Steinhauer, Jennifer. 2020. {``Veterans {Fortify} the {Ranks} of
{Militias Aligned With Trump}'s {Views}.''} \emph{The New York Times:
U.S.}
\url{https://www.nytimes.com/2020/09/11/us/politics/veterans-trump-protests-militias.html}.

\bibitem[\citeproctext]{ref-stewart2022}
Stewart, Charles, III. 2022. {``Trust in {Elections}.''} \emph{Daedalus}
151(4): 234--53.
doi:\href{https://doi.org/10.1162/daed_a_01953}{10.1162/daed\_a\_01953}.

\bibitem[\citeproctext]{ref-vanbavel2021}
Van Bavel, Jay J., and Dominic J. Packer. 2021. \emph{The Power of Us :
Harnessing Our Shared Identities to Improve Performance, Increase
Cooperation, and Promote Social Harmony}. First edition. New York:
Little, Brown Spark.

\bibitem[\citeproctext]{ref-wang2013}
Wang, Lihshing Leigh, Amber S. Watts, Rawni A. Anderson, and Todd D.
Little. 2013. {``Common {Fallacies} in {Quantitative Research
Methodology}.''} In \emph{The {Oxford Handbook} of {Quantitative
Methods} in {Psychology}: {Vol}. 2: {Statistical Analysis}}, ed. Todd D.
Little. Oxford University Press, 0.
doi:\href{https://doi.org/10.1093/oxfordhb/9780199934898.013.0031}{10.1093/oxfordhb/9780199934898.013.0031}.

\bibitem[\citeproctext]{ref-wetheveterans2022}
We The Veterans. 2022. {``Launch of {Vet} the {Vote}.''}
\url{https://vetthe.vote/blogs/news/launch-of-vet-the-vote}.

\bibitem[\citeproctext]{ref-widaman2022}
Widaman, Keith F., and William Revelle. 2022. {``Thinking Thrice about
Sum Scores, and Then Some More about Measurement and Analysis.''}
\emph{Behavior Research Methods} 55(2): 788--806.
doi:\href{https://doi.org/10.3758/s13428-022-01849-w}{10.3758/s13428-022-01849-w}.

\bibitem[\citeproctext]{ref-widaman2024}
Widaman, Keith F., and William Revelle. 2024. {``Thinking {About Sum
Scores Yet Again}, {Maybe} the {Last Time}, {We Don}'t {Know}, {Oh No} .
. .: {A Comment} On.''} \emph{Educational and Psychological Measurement}
84(4): 637--59.
doi:\href{https://doi.org/10.1177/00131644231205310}{10.1177/00131644231205310}.

\bibitem[\citeproctext]{ref-wire2024}
Wire, Sarah D., Phillip M. Bailey, Mary Jo Pitzl, Trevor Hughes, Erik
Pfantz, John Wisely, and Deborah Barfield Berry. 2024. {``Counting Votes
Is Now a Dangerous Job: How It Feels for Frontline, Swing-State
Workers.''}
\url{https://www.usatoday.com/story/news/politics/elections/2024/10/28/election-workers-2024-hostility/75586254007/}
(November 5, 2024).

\bibitem[\citeproctext]{ref-xiao2016}
Xiao, Y. Jenny, Géraldine Coppin, and Jay J. Van Bavel. 2016.
{``Perceiving the {World Through Group-Colored Glasses}: {A Perceptual
Model} of {Intergroup Relations}.''} \emph{Psychological Inquiry} 27(4):
255--74. \url{https://www.jstor.org/stable/26159704} (October 26, 2024).

\bibitem[\citeproctext]{ref-xiao2012}
Xiao, Y. Jenny, and Jay J. Van Bavel. 2012. {``See {Your Friends Close}
and {Your Enemies Closer}: {Social Identity} and {Identity Threat Shape}
the {Representation} of {Physical Distance}.''} \emph{Personality and
Social Psychology Bulletin} 38(7): 959--72.
doi:\href{https://doi.org/10.1177/0146167212442228}{10.1177/0146167212442228}.

\end{CSLReferences}




\end{document}
